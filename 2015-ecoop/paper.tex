
\let\oldvec\vec% Store \vec in \oldvec
\documentclass{llncs}
\let\vec\oldvec% Restore \vec from \oldvec
\usepackage{llncsdoc}

% % Build
% \usepackage{subfiles}

% % Math
 \usepackage{amsmath}
 \usepackage{amssymb}
 \usepackage{math}
 \usepackage{mathtools}

% % Graphics
 \usepackage{graphicx}

% % Text stuff
 \usepackage{listings}
 \usepackage[T1]{fontenc}
 \usepackage[utf8]{inputenc}
 \usepackage{url}
 \usepackage[usenames]{color}

% % PL Formulas
 \usepackage{latexsym}
 \usepackage{bcprules}
 \usepackage{prooftree}


% % Layout
 \usepackage{xspace}% space in macros

% % Makes tables look beautiful
 \usepackage{booktabs}
 \usepackage{tabularx}

 % Borrowed from dot_macros by Nada Amin
\newcommand{\figurebox}[1]
        {\fbox{\begin{minipage}{\textwidth} #1 \medskip\end{minipage}}}
\newcommand{\twofig}[3]
        {\begin{figure*}[t]#3\ \hrulefill\
        \caption{\label{#1}#2}\end{figure*}}
\newcommand{\boxfig}[3]
        {\begin{figure*}\figurebox{#3\caption{\label{#1}#2}}\end{figure*}}
\newcommand{\figref}[1]
        {Figure~\ref{#1}}

% yin-yang rules
\newcommand{\eval}[2]{\_eval[#1] #2}
\newcommand{\lift}[2]{\_lift[#1]\;#2}
%\newcommand{\transf}[2]{\textlbrackdbl #1 \textrbrackdbl \reduces #2}
\newcommand{\ttransf}[2]{ \llbracket #1 \rrbracket^E \transforms #2 }
\newcommand{\Ttransf}[2]{ \llbracket #1 \rrbracket_T^R \transforms #2 }
% %\newcommand{\yytransf}[2]{\llparenthesis #1 \rrparenthesis_{yy}^Y}
\newcommand{\yytransf}[2]{\left\langle #1 \right\rangle^Y}
\newcommand{\trone}[1]{\left\llbracket #1 \right\rrbracket}
\newcommand{\ttrone}[1]{\llbracket #1 \rrbracket}
\newcommand{\yytr}[1]{\left\llbracket #1 \right\rrbracket}
\newcommand{\Ttrone}[1]{\llbracket #1 \rrbracket}
%\newcommand{\yy}[2]{\left\langle #1 \right\rangle_{yy}^{#2}}
\newcommand{\yy}[2]{\llparenthesis #1 \rrparenthesis^{#2}}
\newcommand{\yytrone}[1]{\yy{#1}{Y}}
%\newcommand{\macro}[1]{\left\langle #1 \right\rangle_x^v}
\newcommand{\macro}[1]{\llparenthesis #1 \rrparenthesis^v}
%\newcommand{\Func}[2]{#1 \rightarrow #2}
\newcommand{\Func}[2]{#1 \Rightarrow #2} % Using Scala notation!
\newcommand{\func}[2]{\lambda#1. #2}
\newcommand{\poly}[2]{\lambda#1. #2}
\newcommand{\univ}[2]{\forall#1. #2}
\newcommand{\exist}[2]{\exists#1. #2}
%\newcommand{\llet}[3]{let\; #1 = #2 \; in \; #3}
\newcommand{\llet}[3]{\klet #1 = #2 \kin \; #3}
\newcommand{\blet}[3]{\klet #1 = #2 \kin \; #3}
\newcommand{\tctx}[2]{#1 \vdash #2}
\newcommand{\yctx}[2]{#1 \vdash #2}
\newcommand{\Rep}[1]{R[#1]}
\newcommand{\dpow}[2]{\llparenthesis #1 \rrparenthesis^{(#2, d_{pow}, d_{dpow}, R, \{Int\})}}
\newcommand{\infmacro}[4]{\infrule[#1]{#2}{\yytrone{#3} \rightsquigarrow #4}}
%\newcommand{\infyy}[4]{\infax[#1]{\trone{\frac{\typesetax{#2}}{\typesetax{#3}}} = #4}}
%\newcommand{\infyyax}[3]{\infax[#1]{\trone{\typesetax{#2}} = #3}}
%\newcommand{\infyy}[4]{\infrule[#1]{#2}{\trone{#3} \rightsquigarrow #4}}
\newcommand{\infyy}[4]{\infrule[#1]{#2}{\trone{#3} = #4}}
%\newcommand{\infyyax}[3]{\infax[#1]{\trone{\typesetax{#2}} \rightsquigarrow #3}}
\newcommand{\infyyax}[3]{\infax[#1]{\trone{\typesetax{#2}} = #3}}
\newcommand{\cod}{\operatorname{cod}}
\newcommand{\dom}{\operatorname{dom}}
\newcommand{\vars}[1]{\operatorname{vars(#1)}}
\newcommand{\trule}[1]{#1}

% typing rules (not used here)
\newcommand{\ttag}[1]{\mbox{\textsc{\small(#1)}}}
\newcommand{\infer}[3]{\mbox{#1 }\ba{c} #2 \\ \hline #3 \ea}
\newcommand{\irule}[2]{{\renewcommand{\arraystretch}{1.2}\ba{c} #1
                        \\ \hline #2 \ea}}
\newlength{\trulemargin}
\newlength{\trulewidth}
\newlength{\srulewidth}
\setlength{\trulemargin}{1.75cm}
\setlength{\trulewidth}{83.7mm}
\setlength{\srulewidth}{6.0cm}
\newenvironment{trules}{$\vspace{0.5em}\ba{p{\trulemargin}@{~}p{\trulewidth}@{~}p{\trulemargin}}}{\ea$}
\newenvironment{srules}{$\vspace{0.5em}\ba{p{\trulemargin}@{~}p{\srulewidth}}}{\ea$}
\newcommand{\laxiom}[2]{\ttag{#1} & $ #2 \hfill\ }
\newcommand{\raxiom}[2]{\hfill #2 $& \hfill \ttag{#1}}
\newcommand{\caxiom}[2]{\ttag{#1} & $\hfill #2 \hfill $& \ }
\newcommand{\lrule}[3]{\laxiom{#1}{\irule{#2}{#3}}}
\newcommand{\rrule}[3]{\raxiom{#1}{\irule{#2}{#3}}}
\newcommand{\crule}[3]{\caxiom{#1}{\irule{#2}{#3}}}
\newcommand{\lsrule}[3]{\lsaxiom{#1}{\irule{#2}{#3}}}
\newcommand{\rsrule}[3]{\rsaxiom{#1}{\irule{#2}{#3}}}
\newcommand{\nl}{\end{trules}\\[0.5em] \begin{trules}}
\newcommand{\snl}{\end{srules}\\[0.5em] \begin{srules}}

% commas and semicolons
\newcommand{\comma}{,\,}
\newcommand{\commadots}{\comma \ldots \comma}
\newcommand{\semi}{;\mbox{;};}
\newcommand{\semidots}{\semi \ldots \semi}

% spacing
\newcommand{\gap}{\quad\quad}
\newcommand{\biggap}{\quad\quad\quad}
\newcommand{\nextline}{\\ \\}
\newcommand{\htabwidth}{0.5cm}
\newcommand{\tabwidth}{1cm}
\newcommand{\htab}{\hspace{\htabwidth}}
\newcommand{\tab}{\hspace{\tabwidth}}
\newcommand{\linesep}{\ \hrulefill \ \smallskip}

% math stuff
\newenvironment{myproof}{{\em Proof:}}{$\Box$}
\newenvironment{proofsketch}{{\em Proof Sketch:}}{$\Box$}
\newcommand{\Case}{{\em Case\ }}

% make ; a delimiter in math mode
% \mathcode`\;="8000 % Makes ; active in math mode
% {\catcode`\;=\active \gdef;{\;}}
% \mathchardef\semicolon="003B

% reserved words
\newcommand{\mathem}{\bf}

% brackets
\newcommand{\set}[1]{\{#1\}}
\newcommand{\sbs}[1]{\lquote #1 \rquote}

% arrays
\newcommand{\ba}{\begin{array}}
\newcommand{\ea}{\end{array}}
\newcommand{\bda}{\[\ba}
\newcommand{\eda}{\ea\]}
\newcommand{\ei}{\end{array}}
\newcommand{\bcases}{\left\{\begin{array}{ll}}
\newcommand{\ecases}{\end{array}\right.}

% \cal ids
\renewcommand{\AA}{{\cal A}}
\newcommand{\BB}{{\cal B}}
\newcommand{\CC}{{\cal C}}
\newcommand{\DD}{{\cal D}}
\newcommand{\EE}{{\cal E}}
\newcommand{\FF}{{\cal F}}
\newcommand{\GG}{{\cal G}}
\newcommand{\HH}{{\cal H}}
\newcommand{\II}{{\cal I}}
\newcommand{\JJ}{{\cal J}}
\newcommand{\KK}{{\cal K}}
\newcommand{\LL}{{\cal L}}
\newcommand{\MM}{{\cal M}}
\newcommand{\NN}{{\cal N}}
\newcommand{\OO}{{\cal O}}
\newcommand{\PP}{{\cal P}}
\newcommand{\QQ}{{\cal Q}}
\newcommand{\RR}{{\cal R}}
\newcommand{\TT}{{\cal T}}
\newcommand{\UU}{{\cal U}}
\newcommand{\VV}{{\cal V}}
\newcommand{\WW}{{\cal W}}
\newcommand{\XX}{{\cal X}}
\newcommand{\YY}{{\cal Y}}
\newcommand{\ZZ}{{\cal Z}}

% misc symbols
\newcommand{\dhd}{\!\!\!\!\!\rightarrow}
\newcommand{\Dhd}{\!\!\!\!\!\Rightarrow}
\newcommand{\la}{\langle}
\newcommand{\ra}{\rangle}
\newcommand{\eg}{{\em e.g.}}

% misc identifiers
\newcommand{\fn}{\mbox{\sl fn}}
\newcommand{\bn}{\mbox{\sl bn}}
\newcommand{\sig}{\mbox{\sl sig}}
\newcommand{\IF}{\mbox{\mathem if}}
\newcommand{\OTHERWISE}{\mbox{\mathem otherwise}}
\newcommand{\expand}{\prec}
\newcommand{\weakexpand}{\prec^W}
\newcommand{\spcomma}{~,~}

%\newcommand{\inst}{\mbox{\mathem inst}}
\newcommand{\trans}[1]{\la\!\la#1\ra\!\ra}
%\newcommand{\J}{\justifies}
%\newcommand{\U}{\using}

% names
\newcommand{\Scala}{\mbox{\textsc{Scala}}}
\newcommand{\Java}{\mbox{\textsc{Java}}}
\newcommand{\sedsl}{Direct \edsl}
\newcommand{\sedsls}{Direct \edsls}
\newcommand{\dedsl}{Deep \edsl}
\newcommand{\dedsls}{Deep \edsls}
\newcommand{\edsl}{EDSL\xspace}
\newcommand{\edsls}{EDSLs\xspace}
\newcommand{\THE}{the}
\newcommand{\A}{a}


%\renewcommand\textfraction{.05}
%\renewcommand\floatpagefraction{.9}
%\renewcommand\topfraction{.8}

%%%%%%%%%%%%%%%%%%%%%%%%%%%%%%%%%%%%%%%
%   Language abstraction commands     %
%%%%%%%%%%%%%%%%%%%%%%%%%%%%%%%%%%%%%%%

%% Relations
% Subtype
\newcommand{\sub}{<:}
% Type assignment
\newcommand{\typ}{:}
\newcommand{\approxtyp}{:_{<:}}
% reduction
\newcommand{\reduces}{\;\rightarrow\;}
\newcommand{\transforms}{\;\rightsquigarrow \;}
% well-formedness
\newcommand{\wf}{\;\mbox{\textbf{wf}}}
\newcommand{\wfe}{\;\mbox{\textbf{wfe}}}

%% Operators
% Type selection
\newcommand{\tsel}{\#}
% Function type
\newcommand{\tfun}{\rightarrow}
\newcommand{\dfun}[3]{(#1\!:\!#2) \Rightarrow #3}
% Conjunction
\newcommand{\tand}{\wedge}
% Disjunction
\newcommand{\tor}{\vee}
% Singleton type suffix
\newcommand{\sing}{.\textbf{type}}

%% Syntax
% Header for typing rules
\newcommand{\judgement}[2]{{\bf #1} \hfill #2}
% Refinement
\newcommand{\refine}[2]{\left\{#1 \Rightarrow #2 \right\}}
% Field definitions
\newcommand{\ldefs}[1]{\left\{#1\right\}}
% Member sequences
\newcommand{\seq}[1]{\overline{#1}}
% Lambda
\newcommand{\dabs}[3]{(#1\!:\!#2)\Rightarrow #3}
\newcommand{\abs}[3]{\lambda #1\!:\!#2.#3}
% Application
\newcommand{\app}[2]{#1\;#2}
% Method Application
\newcommand{\mapp}[3]{#1.#2(#3)}
% Substitution
\newcommand{\subst}[3]{[#1/#2]#3}
% Object creation
\newcommand{\new}[3]{\textbf{val }#1 = \textbf{new }#2 ;\; #3}
\newcommand{\anfmapp}[5]{\textbf{val }#1 = {#2.#3(#4)} ;\; #5}
\newcommand{\anfmexe}[5]{\textbf{val }#1 = {#2.#3\ldots\;#4} ;\; #5}
%\renewcommand{\new}[3]{#1 \leftarrow #2 \,\textbf{in}\, #3}
% Field declaration
\newcommand{\Ldecl}[3]{#1 : #2..#3}%{#1 \operatorname{>:} #2 \operatorname{<:} #3}
\newcommand{\ldecl}[2]{#1 : #2}
\newcommand{\mdecl}[3]{#1 : #2 \tfun #3}
% Top and Bottom
\newcommand{\Top}{\top}%{\textbf{Top}}
\newcommand{\Bot}{\bot}%\textbf{Bot}}
% Environment extension
%\newcommand{\envplus}[1]{\uplus \{ #1 \}}
\newcommand{\envplus}[1]{, #1}
% Reduction
\newcommand{\reduction}[2]{#1 \reduces #2 }
\DeclareMathOperator{\klet}{\mathbf{let}} % Used in examples
\DeclareMathOperator{\tlet}{\mathbf{let}} % Used in text


\DeclareMathOperator{\kin}{\mathbf{in}} % Used in examples
\DeclareMathOperator{\tin}{\mathbf{in}} % Used in text

\DeclareMathOperator{\kfix}{\mathbf{fix}}
\DeclareMathOperator{\tfix}{\mathbf{fix}} % Used in text
\DeclareMathOperator{\kif}{\mathbf{if}}
\DeclareMathOperator{\kthen}{\mathbf{then}}
\DeclareMathOperator{\kelse}{\mathbf{else}}
\DeclareMathOperator{\kint}{\mathbf{Int}}
\DeclareMathOperator{\kbool}{\mathbf{Bool}}
\DeclareMathOperator{\ktrue}{\mathbf{true}}
\DeclareMathOperator{\kfalse}{\mathbf{false}}
%\newcommand{\btyp}{B}
\newcommand{\btyp}{\iota}

\newcommand{\calculus}{the calculus\xspace}


% ----- begin macros
\lstdefinelanguage{Scala}%
{morekeywords={abstract,%
  case,catch,char,class,%
  def,else,extends,final,for,%
  if,import,implicit,%
  match,module,%
  new,null,%
  object,override,%
  %package,% commented out for a specific example
  private,protected,public,%
  for,public,return,super,%
  this,throw,trait,try,type,%
  val,var,%
  with,while,%
  yield,%
  macro%
  },%
  sensitive,%
  morecomment=[l]//,%
  morecomment=[s]{/*}{*/},%
  morestring=[b]",%
  morestring=[b]',%
  showstringspaces=false%
}[keywords,comments,strings]%

\lstset{language=Scala,%
  mathescape=true,%
%  columns=[c]fixed,%
%  basewidth={0.5em, 0.40em},%
  aboveskip=1pt,%\smallskipamount,
  belowskip=1pt,%\negsmallskipamount,
  lineskip=-0.2pt,
  basewidth={0.54em, 0.4em},%
%  basicstyle=\ttfamily,%\scriptsize,%
  basicstyle=\footnotesize\ttfamily,
  keywordstyle=\sffamily\bfseries%
%  keywordstyle=\sffamily\bfseries,%
%  xleftmargin=0.5cm
}


\newcommand{\commentstyle}[1]{\slseries{#1}}
\newcommand{\keywordstyle}[1]{\bfseries{#1}}

% Code
\lstnewenvironment{listing}{\lstset{language=Scala}}{}
\lstnewenvironment{listingsmall}{\lstset{language=Scala,basicstyle=\small\ttfamily}}{}
\lstnewenvironment{listingtiny}{\lstset{language=Scala,basicstyle=\scriptsize\ttfamily}}{}

\newcommand{\scode}[1]{\lstinline[language=Scala,columns=fixed,basicstyle=\ttfamily,keywordstyle=\ttfamily]|#1|}
\newcommand{\jcode}[1]{\lstinline[language=Java,flexiblecolumns=true,basicstyle=\ttfamily]{#1}}

\newcommand{\code}[1]{\scode{#1}}
\newcommand{\sct}[1]{(\S \ref{#1})}

% TODOs:
\newif\ifshowTodos
\showTodostrue
%\showTodosfalse % Uncomment this to hide all TODOs.
\ifshowTodos
\newcommand{\todo}[1]{{\color{red} \textbf{[TODO: #1 ]}}}
\else
\newcommand{\todo}[1]{}
\fi
% paper specific commands


\begin{document}

\title{Scala Inline}

\maketitle

\begin{abstract}
\cite{scala_macros}
\end{abstract}
% \category{D.3.3}{Programming Languages}{Language Constructs and Features}

\keywords
Partial Evaluation, Macros


t\clearpage
\section{The \calculus Calculus}
\label{sct:calculus}

\begin{figure}[H]
\begin{multicols}{2}
\syntaxfig{
  S,\ T,\ U ::=                     & \lindent{\mbox{Types:}}              \\
  \gap \i{S} \ra \j{T}              & \mbox{function type}                 \\
  \gap \{ \seq{x: \i{S}} \}         & \mbox{record type}                   \\
  \gap [X <: \i{S}] \ra \j{T}       & \mbox{universal type}                \\
  \gap Any                          & \mbox{top type}                      \\
  \i{T},\ \j{T},\ \k{T},\ \l{T} ::= & \lindent{\mbox{Binding-Time Types:}} \\
  \gap X                            & \mbox{type identifier}               \\
  \gap \dynamic{T}                  & \mbox{dynamic type}                  \\
  \gap \inline{T}                   & \mbox{inline type}                   \\
  \Gamma ::=                        & \lindent{\mbox{Contexts:}}           \\
  \gap \emptyset                    & \mbox{empty context}                 \\
  \gap \Gamma,\ x: \i{T}            & \mbox{term binding}                  \\
  \gap \Gamma,\ X <: \i{T}          & \mbox{type binding}                  \\
}
\syntaxfig{
  t ::=                             & \lindent{\mbox{Terms:}}              \\
  \gap x,\ y                        & \mbox{identifier}                    \\
  \gap \i{v}                        & \mbox{literal}                       \\
  \gap \dynamic{t}                  & \mbox{dynamic coercion}              \\
  \gap t(t)                         & \mbox{application}                   \\
  \gap t.x                          & \mbox{selection}                     \\
  \gap t [\i{T}]                    & \mbox{type application}              \\
  \i{v} ::=                         & \mbox{Literals:}                     \\
  \gap v                            & \mbox{dynamic literal}               \\
  \gap \inline{v}                   & \mbox{inline literal}                \\
  v ::=                             & \lindent{\mbox{Values:}}             \\
  \gap (x: \i{T}) \ra t             & \mbox{function value}                \\
  \gap \{ \seq{x = t} \}            & \mbox{record value}                  \\
  \gap [X <: \i{T}] \ra t           & \mbox{type abstraction value}        \\
}
\end{multicols}
\caption{Syntax of \calculus}
\end{figure}

We formalize the essence of our partial evaluation system in a minimalistic calculus
based on kernel \fsub \cite{tapl} with lazy records. To accommodate predictable partial evaluation we
introduce binding-time annotations into the type system as types that
represent two kinds of bindings:

\begin{enumerate}
  \item \textbf{Dynamic}. Corresponds to terms that are expected to be evaluted at runtime.
  \item \textbf{Inline}. Corresponds to terms that must be evaluated at compile-time.
\end{enumerate}

To simplify judgments in our formalization we use concise $\i{T}$ syntax to abstract over
binding times of types. Here $i$ signifies the bit of information that says if type is inline
or not and $T$ carries underlying type that is being annotated. So for example in $\inline{Any}$
we get $i = inline$ and $T = Any$.

Similarly we abstract over binding times of literals through $\i{v}$ notation that has analogous
interpretation to the one we use for types.

\subsection{Type composition}

\subsection{Well-formed types}

Even though binding-time information is represented as types, not all
of the possible combinations of types and binding-times is correct.
We restrict types to disallow nesting of more specific binding times
into less specific ones.

\begin{figure}[H]
  \infax[\textsc{W-Any}]
  {\wft{\i{Any}}}

  \infrule[\textsc{W-Abs}]
  {i \leq j \ \ \ \ i \leq k \ \ \ \ \wft{\j{T_1}} \ \ \ \ \wft{\k{T_2}}}
  {\wft{\i{(\j{T_1} \ra \k{T_2})}}}

  \infrule[\textsc{W-TAbs}]
  {i \leq j \ \ \ \ i \leq k \ \ \ \ \wft{[X \mapsto \j{S}] \k{T}}}
  {\wft{\i{([X <: \j{S}] \ra \k{T})}}}

  \infrule[\textsc{W-Rec}]
  {i \leq \seq{j} \ \ \ \ \seq{\wft{\j{T}}}}
  {\wft{\i{\{\seq{x: \j{T}}\}}}}
\caption{Well-formed types $\wft{\i{T}}$.}
\end{figure}

We represent notion of more specific binding-times through
a simple partial order on binding time annotations.

\begin{figure}[H]
  \begin{center}
  $ dynamic \leq dynamic $ \\
  $ inline  \leq dynamic $ \\
  $ inline  \leq inline  $
  \end{center}
\caption{Partial order on binding-time $i \leq j$}
\end{figure}

This restriction allows us to reject programs that have inconsistent binding-time annotations.
For example the following function has incorrectly annotated parameter binding time:
\begin{equation}\nonumber
    (x: \inline{Int}) \ra x + 1
\end{equation}

This is inconsistent because a dynamic function may not have any non-dynamic parameters.
As described in \textsc{W-Abs} functions may only have parameters that are at most as specific
as function binding-time. In our example this doesn't hold as $inline$ is more
specific than $dynamic$.

\subsection{Subtyping}

\calculus integrates binding-time annotation into subtyping relation on regular types by
threading inlining information throughout all of the standard subtyping rules.

\begin{figure}[H]
  \infax[\textsc{S-Top}]
  {\Gamma \ts \i{S} <: \i{Any}}

  \infax[\textsc{S-Refl}]
  {\Gamma \ts \i{S} <: \i{S}}

  \infrule[\textsc{S-Trans}]
  {\Gamma \ts \i{S} <: \j{U} \ \ \ \ \Gamma \ts \j{U} <: \k{T}}
  {\Gamma \ts \i{S} <: \j{U}}

  \infrule[\textsc{S-Arrow}]
  {\Gamma \ts \k{T_1} <: \i{S_1} \ \ \ \ \Gamma \ts \j{S_2} <: \l{T_2}}
  {\Gamma \ts \i{S_1} \ra \j{S_2} <: \k{T_1} \ra \l{T_2}}

  \infrule[\textsc{S-All}]
  {\Gamma,\ X <: \i{U_1} \ts \j{S_2} <: \k{T_2}}
  {\Gamma \ts [X <: \i{U_1}] \ra \j{S_2} <: [X <: \i{U_1}] \ra \k{T_2}}

  \infrule[\textsc{S-Perm}]
  {\{ x_p : i_p S_p\ ^{p \in 1..n}\} $ is permutation of $ \{ y_p: j_p T_p\ ^{p \in 1..n} \}}
  {\Gamma \ts \{ x_p : i_p S_p\ ^{p \in 1..n}\} <: \{ y_p: j_p T_p\ ^{p \in 1..n} \} }

  \infrule[\textsc{S-Depth}]
  {\forall p \in 1..n.\ \Gamma \ts i_p S_p <: j_p T_p}
  {\Gamma \ts \{ x_p: i_p S_p\ ^{p \in 1..n}\} <: \{ x_p: j_p T_p\ ^{p \in 1..n} \}}

  \infax[\textsc{S-Width}]
  {\Gamma \ts \{ x_p: i_p T_p\ ^{p \in 1..n+m} \} <: \{ x_p: i_p T_p\ ^{p \in 1..n} \}}
\caption{Subtyping $\Gamma \ts T_1 <: T_2$.}
\end{figure}

Apart from that we also introduce subtyping on binding-time types.

\begin{figure}[H]
  \infrule[\textsc{S-TVar}]
  {X <: \i{T} \in \Gamma}
  {\Gamma \ts X <: \i{T}}

  \infrule[\textsc{S-Inline}]
  {i = j \ \ \ \ \Gamma \ts S <: T}
  {\Gamma \ts \i{S} <: \j{T}}
\caption{Subtyping of binding-time types $\Gamma \ts \i{T_1} <: \j{T_2}$.}
\end{figure}

Two binding-time types are subtypes if their underlying types are
subtypes and if they have the same binding time.

\subsection{Type polymorphism}

Our system retains traditional type abstraction means inherited from \fsub.
We extend it to accomodate encoding of binding-times into types. This allows
us to specify binding type of the abstracted generic type:
\begin{equation}\nonumber
  [T <: \dynamic{Any}] \ra (x: T) \ra x
\end{equation}

For this particular identity function we need to restrict subset of all admissible
types to only allow $dynamic$ ones. Passing an $inline$ type would not make sense
as the resulting type would have not been well-formed.

\subsection{Typing}

\begin{figure}[H]
  \infrule[\textsc{T-Ident}]
  {x: \i{T} \in \Gamma}
  {\Gamma \ts x: \i{T}}

  \infrule[\textsc{T-Rec}]
  {\Gamma \ts \seq{t}: \seq{\j{T}} \ \ \ \ \wft{\i{\{\seq{x: \j{T}}\}}}}
  {\Gamma \ts \i{\{ \seq{x = t} \}} : \i{\{\seq{x: \j{T}}\}}}

  \infrule[\textsc{T-App}]
  {\Gamma \ts t_1: \i{(\j{T_1} \ra \k{T_2})} \ \ \ \ \Gamma \ts t_2: \j{T_1}}
  {\Gamma \ts t_1 (t_2) : \k{T_2}}

  \infrule[\textsc{T-Sel}]
  {\Gamma \ts t: \i{\{ x = \j{T_1}, \seq{y = \k{T_2}} \}}}
  {\Gamma \ts t.x : \j{T_1}}

  \infrule[\textsc{T-Sub}]
  {\Gamma \ts t: \i{S} \ \ \ \ \Gamma \ts \i{S} <: \j{T}}
  {\Gamma \ts t: \j{T}}

  \infrule[\textsc{T-Abs}]
  {\Gamma,\ x: \j{T_1} \ts t: \k{T_2} \ \ \ \ \wft{\i{(\j{T_1} \ra \k{T_2})}}}
  {\Gamma \ts \i{((x: \j{T_1}) \ra t)} : \i{(\j{T_1} \ra \k{T_2})}}

  \infrule[\textsc{T-TAbs}]
  {\Gamma,\ X <: \j{T_0} \ts t_2: \k{T_2} \ \ \ \ \wft{\i{([X <: \j{T_1}] \ra \k{T_2})}}}
  {\Gamma \ts \i([X <: \j{T_1}] \ra t_2): \i{([X <: T_1] \ra \k{T_2})}}

  \infrule[\textsc{T-TApp}]
  {\Gamma \ts t: \i{([X <: \j{T_1}] \ra \k{T_2})}  \ \ \ \ \Gamma \ts \l{T} <: \j{T_1}}
  {\Gamma \ts t[\l{T}] : [X \mapsto \l{T}] \k{T_2}}

  \infrule[\textsc{T-Dynamic}]
  {\Gamma \ts t: \inline{T}}
  {\Gamma \ts \dynamic{t} : \dynamic{T}}
\caption{Typing $\Gamma \ts t: \i{T}$.}
\end{figure}

Similarly to the changes made to the subtyping relation we thread binding-time information
throughout typing relation. Apart from that we also ensure that all literals produced by the
user have well-formed types.

\subsection{Partial Evaluation}

\begin{figure}[H]
  \infrule[\textsc{PE-Abs}]
  {t \pe t'}
  {(x: \i{T}) \ra t \pe (x: T) \ra t'}

  \infrule[\textsc{PE-Rec}]
  {\seq{t} \pe \seq{t'}}
  {\{\seq{x = t}\} \pe \{\seq{x = t'}\}}

  \infrule[\textsc{PE-TAbs}]
  {t \pe t'}
  {[X <: \i{T}] \ra t \pe [X <: \i{T}] \ra t'}

  \infrule[\textsc{PE-App}]
  {t_1 \pe t_1' \ \ \ \ t_1 \ne \inline{t_3} \ \ \ \ t_2 \pe t_2'}
  {t_1(t_2) \pe t_1'(t_2')}

  \infrule[\textsc{PE-Sel}]
  {t \pe t' \ \ \ \ t' \ne \inline{t_3}}
  {t.x \pe t'.x}

  \infrule[\textsc{PE-TApp}]
  {t_1 \pe t_1' \ \ \ \ t_1' \ne \inline{t_3}}
  {t_1[\i{T_1}] \pe t_1'[\i{T_1}]}

  \infrule[\textsc{PE-InlineApp}]
  {t_1 \pe \inline{(x: \i{T}) \ra t} \ \ \ \ t_2 \pe t_2' \ \ \ \ [x \mapsto t_2'] t \pe t'}
  {t_1(t_2) \pe t'}

  \infrule[\textsc{PE-InlineSel}]
  {t \pe \inline{\{x = t_x,\ \seq{y = t_y}\}} \ \ \ \ t_x \pe t_x'}
  {t.x \pe t_x'}

  \infrule[\textsc{PE-InlineTApp}]
  {t_1 \pe \inline [X <: \i{T_2}] \ra t_2 \ \ \ \ [X \mapsto \i{T_1}] t_2 \pe t_2'}
  {t_1[\i{T_1}] \pe t_2'}

  \infax[\textsc{PE-InlineValue}]
  {\inline{v} \pe \inline{v}}

  \infrule[\textsc{PE-Dynamic}]
  {t \pe \inline{t'} \ \ \ \ t' \pe t''}
  {\dynamic{t} \pe t''}
\caption{Partial evaluation $t \pe t'$}
\label{fig:partial-evaluation}
\end{figure}

\subsection{Evaluation}

Once partial evaluation is complete we strip all binding-time annotations on types
and convert inline terms into corresponding dynamic ones. After that we can use
standard \fsub evaluation rules augmented with lazy records semantics (\textsc{E-Sel}).

\begin{figure}[H]
  \infax[\textsc{E-Value}]
  {v \e v}

  \infrule[\textsc{E-App}]
  {t_1 \e (x: T) \ra t \ \ \ \ t_2 \e v \ \ \ \ [x \mapsto v] t \e v'}
  {t_1(t_2) \e v'}

  \infrule[\textsc{E-TApp}]
  {t_1 \e [X <: T_2] \ra t_2 \ \ \ \ [X \mapsto T_1] t_2 \e v}
  {t_1[T_1] \e v}

  \infrule[\textsc{E-Sel}]
  {t \e \{ x = t_x,\ \seq{y = t_y}\} \ \ \ \ t_x \e v}
  {t.x \e v}
\caption{Evaluation $t \e v$}
\end{figure}

\subsection{Conjectures}

\begin{enumerate}
  \item Progress and preservation of partial evaluation.
  \item Progress and preservation of evaluation.
\end{enumerate}

\bibliographystyle{plain}

\bibliography{vjovanov-lib}

\end{document}



