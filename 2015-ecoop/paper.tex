
\let\oldvec\vec% Store \vec in \oldvec
\documentclass{llncs}
\let\vec\oldvec% Restore \vec from \oldvec
\usepackage{llncsdoc}

% % Build
% \usepackage{subfiles}

% % Math
 \usepackage{amsmath}
 \usepackage{amssymb}
 \usepackage{math}
 \usepackage{mathtools}

% % Graphics
 \usepackage{graphicx}
 \usepackage{tikz}

% % Text stuff
 \usepackage{listings}
 \usepackage[T1]{fontenc}
 \usepackage[utf8]{inputenc}
 \usepackage{url}
 % \usepackage[usenames]{color}

% % PL Formulas
 \usepackage{latexsym}
 \usepackage{bcprules}
 \usepackage{prooftree}

% % Layout
 \usepackage{xspace}% space in macros
 \usepackage{float}

% % Makes tables look beautiful
 \usepackage{booktabs}
 \usepackage{tabularx}

 % Borrowed from dot_macros by Nada Amin
\newcommand{\figurebox}[1]
        {\fbox{\begin{minipage}{\textwidth} #1 \medskip\end{minipage}}}
\newcommand{\twofig}[3]
        {\begin{figure*}[t]#3\ \hrulefill\
        \caption{\label{#1}#2}\end{figure*}}
\newcommand{\boxfig}[3]
        {\begin{figure*}\figurebox{#3\caption{\label{#1}#2}}\end{figure*}}
\newcommand{\figref}[1]
        {Figure~\ref{#1}}

% yin-yang rules
\newcommand{\eval}[2]{\_eval[#1] #2}
\newcommand{\lift}[2]{\_lift[#1]\;#2}
%\newcommand{\transf}[2]{\textlbrackdbl #1 \textrbrackdbl \reduces #2}
\newcommand{\ttransf}[2]{ \llbracket #1 \rrbracket^E \transforms #2 }
\newcommand{\Ttransf}[2]{ \llbracket #1 \rrbracket_T^R \transforms #2 }
% %\newcommand{\yytransf}[2]{\llparenthesis #1 \rrparenthesis_{yy}^Y}
\newcommand{\yytransf}[2]{\left\langle #1 \right\rangle^Y}
\newcommand{\trone}[1]{\left\llbracket #1 \right\rrbracket}
\newcommand{\ttrone}[1]{\llbracket #1 \rrbracket}
\newcommand{\yytr}[1]{\left\llbracket #1 \right\rrbracket}
\newcommand{\Ttrone}[1]{\llbracket #1 \rrbracket}
%\newcommand{\yy}[2]{\left\langle #1 \right\rangle_{yy}^{#2}}
\newcommand{\yy}[2]{\llparenthesis #1 \rrparenthesis^{#2}}
\newcommand{\yytrone}[1]{\yy{#1}{Y}}
%\newcommand{\macro}[1]{\left\langle #1 \right\rangle_x^v}
\newcommand{\macro}[1]{\llparenthesis #1 \rrparenthesis^v}
%\newcommand{\Func}[2]{#1 \rightarrow #2}
\newcommand{\Func}[2]{#1 \Rightarrow #2} % Using Scala notation!
\newcommand{\func}[2]{\lambda#1. #2}
\newcommand{\poly}[2]{\lambda#1. #2}
\newcommand{\univ}[2]{\forall#1. #2}
\newcommand{\exist}[2]{\exists#1. #2}
%\newcommand{\llet}[3]{let\; #1 = #2 \; in \; #3}
\newcommand{\llet}[3]{\klet #1 = #2 \kin \; #3}
\newcommand{\blet}[3]{\klet #1 = #2 \kin \; #3}
\newcommand{\tctx}[2]{#1 \vdash #2}
\newcommand{\yctx}[2]{#1 \vdash #2}
\newcommand{\Rep}[1]{R[#1]}
\newcommand{\dpow}[2]{\llparenthesis #1 \rrparenthesis^{(#2, d_{pow}, d_{dpow}, R, \{Int\})}}
\newcommand{\infmacro}[4]{\infrule[#1]{#2}{\yytrone{#3} \rightsquigarrow #4}}
%\newcommand{\infyy}[4]{\infax[#1]{\trone{\frac{\typesetax{#2}}{\typesetax{#3}}} = #4}}
%\newcommand{\infyyax}[3]{\infax[#1]{\trone{\typesetax{#2}} = #3}}
%\newcommand{\infyy}[4]{\infrule[#1]{#2}{\trone{#3} \rightsquigarrow #4}}
\newcommand{\infyy}[4]{\infrule[#1]{#2}{\trone{#3} = #4}}
%\newcommand{\infyyax}[3]{\infax[#1]{\trone{\typesetax{#2}} \rightsquigarrow #3}}
\newcommand{\infyyax}[3]{\infax[#1]{\trone{\typesetax{#2}} = #3}}
\newcommand{\cod}{\operatorname{cod}}
\newcommand{\dom}{\operatorname{dom}}
\newcommand{\vars}[1]{\operatorname{vars(#1)}}
\newcommand{\trule}[1]{#1}

% typing rules (not used here)
\newcommand{\ttag}[1]{\mbox{\textsc{\small(#1)}}}
\newcommand{\infer}[3]{\mbox{#1 }\ba{c} #2 \\ \hline #3 \ea}
\newcommand{\irule}[2]{{\renewcommand{\arraystretch}{1.2}\ba{c} #1
                        \\ \hline #2 \ea}}
\newlength{\trulemargin}
\newlength{\trulewidth}
\newlength{\srulewidth}
\setlength{\trulemargin}{1.75cm}
\setlength{\trulewidth}{83.7mm}
\setlength{\srulewidth}{6.0cm}
\newenvironment{trules}{$\vspace{0.5em}\ba{p{\trulemargin}@{~}p{\trulewidth}@{~}p{\trulemargin}}}{\ea$}
\newenvironment{srules}{$\vspace{0.5em}\ba{p{\trulemargin}@{~}p{\srulewidth}}}{\ea$}
\newcommand{\laxiom}[2]{\ttag{#1} & $ #2 \hfill\ }
\newcommand{\raxiom}[2]{\hfill #2 $& \hfill \ttag{#1}}
\newcommand{\caxiom}[2]{\ttag{#1} & $\hfill #2 \hfill $& \ }
\newcommand{\lrule}[3]{\laxiom{#1}{\irule{#2}{#3}}}
\newcommand{\rrule}[3]{\raxiom{#1}{\irule{#2}{#3}}}
\newcommand{\crule}[3]{\caxiom{#1}{\irule{#2}{#3}}}
\newcommand{\lsrule}[3]{\lsaxiom{#1}{\irule{#2}{#3}}}
\newcommand{\rsrule}[3]{\rsaxiom{#1}{\irule{#2}{#3}}}
\newcommand{\nl}{\end{trules}\\[0.5em] \begin{trules}}
\newcommand{\snl}{\end{srules}\\[0.5em] \begin{srules}}

% commas and semicolons
\newcommand{\comma}{,\,}
\newcommand{\commadots}{\comma \ldots \comma}
\newcommand{\semi}{;\mbox{;};}
\newcommand{\semidots}{\semi \ldots \semi}

% spacing
\newcommand{\gap}{\quad\quad}
\newcommand{\biggap}{\quad\quad\quad}
\newcommand{\nextline}{\\ \\}
\newcommand{\htabwidth}{0.5cm}
\newcommand{\tabwidth}{1cm}
\newcommand{\htab}{\hspace{\htabwidth}}
\newcommand{\tab}{\hspace{\tabwidth}}
\newcommand{\linesep}{\ \hrulefill \ \smallskip}

% math stuff
\newenvironment{myproof}{{\em Proof:}}{$\Box$}
\newenvironment{proofsketch}{{\em Proof Sketch:}}{$\Box$}
\newcommand{\Case}{{\em Case\ }}

% make ; a delimiter in math mode
% \mathcode`\;="8000 % Makes ; active in math mode
% {\catcode`\;=\active \gdef;{\;}}
% \mathchardef\semicolon="003B

% reserved words
\newcommand{\mathem}{\bf}

% brackets
\newcommand{\set}[1]{\{#1\}}
\newcommand{\sbs}[1]{\lquote #1 \rquote}

% arrays
\newcommand{\ba}{\begin{array}}
\newcommand{\ea}{\end{array}}
\newcommand{\bda}{\[\ba}
\newcommand{\eda}{\ea\]}
\newcommand{\ei}{\end{array}}
\newcommand{\bcases}{\left\{\begin{array}{ll}}
\newcommand{\ecases}{\end{array}\right.}

% \cal ids
\renewcommand{\AA}{{\cal A}}
\newcommand{\BB}{{\cal B}}
\newcommand{\CC}{{\cal C}}
\newcommand{\DD}{{\cal D}}
\newcommand{\EE}{{\cal E}}
\newcommand{\FF}{{\cal F}}
\newcommand{\GG}{{\cal G}}
\newcommand{\HH}{{\cal H}}
\newcommand{\II}{{\cal I}}
\newcommand{\JJ}{{\cal J}}
\newcommand{\KK}{{\cal K}}
\newcommand{\LL}{{\cal L}}
\newcommand{\MM}{{\cal M}}
\newcommand{\NN}{{\cal N}}
\newcommand{\OO}{{\cal O}}
\newcommand{\PP}{{\cal P}}
\newcommand{\QQ}{{\cal Q}}
\newcommand{\RR}{{\cal R}}
\newcommand{\TT}{{\cal T}}
\newcommand{\UU}{{\cal U}}
\newcommand{\VV}{{\cal V}}
\newcommand{\WW}{{\cal W}}
\newcommand{\XX}{{\cal X}}
\newcommand{\YY}{{\cal Y}}
\newcommand{\ZZ}{{\cal Z}}

% misc symbols
\newcommand{\dhd}{\!\!\!\!\!\rightarrow}
\newcommand{\Dhd}{\!\!\!\!\!\Rightarrow}
\newcommand{\la}{\langle}
\newcommand{\ra}{\rangle}
\newcommand{\eg}{{\em e.g.}}

% misc identifiers
\newcommand{\fn}{\mbox{\sl fn}}
\newcommand{\bn}{\mbox{\sl bn}}
\newcommand{\sig}{\mbox{\sl sig}}
\newcommand{\IF}{\mbox{\mathem if}}
\newcommand{\OTHERWISE}{\mbox{\mathem otherwise}}
\newcommand{\expand}{\prec}
\newcommand{\weakexpand}{\prec^W}
\newcommand{\spcomma}{~,~}

%\newcommand{\inst}{\mbox{\mathem inst}}
\newcommand{\trans}[1]{\la\!\la#1\ra\!\ra}
%\newcommand{\J}{\justifies}
%\newcommand{\U}{\using}

% names
\newcommand{\Scala}{\mbox{\textsc{Scala}}}
\newcommand{\Java}{\mbox{\textsc{Java}}}
\newcommand{\sedsl}{Direct \edsl}
\newcommand{\sedsls}{Direct \edsls}
\newcommand{\dedsl}{Deep \edsl}
\newcommand{\dedsls}{Deep \edsls}
\newcommand{\edsl}{EDSL\xspace}
\newcommand{\edsls}{EDSLs\xspace}
\newcommand{\THE}{the}
\newcommand{\A}{a}


%\renewcommand\textfraction{.05}
%\renewcommand\floatpagefraction{.9}
%\renewcommand\topfraction{.8}

%%%%%%%%%%%%%%%%%%%%%%%%%%%%%%%%%%%%%%%
%   Language abstraction commands     %
%%%%%%%%%%%%%%%%%%%%%%%%%%%%%%%%%%%%%%%

%% Relations
% Subtype
\newcommand{\sub}{<:}
% Type assignment
\newcommand{\typ}{:}
\newcommand{\approxtyp}{:_{<:}}
% reduction
\newcommand{\reduces}{\;\rightarrow\;}
\newcommand{\transforms}{\;\rightsquigarrow \;}
% well-formedness
\newcommand{\wf}{\;\mbox{\textbf{wf}}}
\newcommand{\wfe}{\;\mbox{\textbf{wfe}}}

%% Operators
% Type selection
\newcommand{\tsel}{\#}
% Function type
\newcommand{\tfun}{\rightarrow}
\newcommand{\dfun}[3]{(#1\!:\!#2) \Rightarrow #3}
% Conjunction
\newcommand{\tand}{\wedge}
% Disjunction
\newcommand{\tor}{\vee}
% Singleton type suffix
\newcommand{\sing}{.\textbf{type}}

%% Syntax
% Header for typing rules
\newcommand{\judgement}[2]{{\bf #1} \hfill #2}
% Refinement
\newcommand{\refine}[2]{\left\{#1 \Rightarrow #2 \right\}}
% Field definitions
\newcommand{\ldefs}[1]{\left\{#1\right\}}
% Member sequences
\newcommand{\seq}[1]{\overline{#1}}
% Lambda
\newcommand{\dabs}[3]{(#1\!:\!#2)\Rightarrow #3}
\newcommand{\abs}[3]{\lambda #1\!:\!#2.#3}
% Application
\newcommand{\app}[2]{#1\;#2}
% Method Application
\newcommand{\mapp}[3]{#1.#2(#3)}
% Substitution
\newcommand{\subst}[3]{[#1/#2]#3}
% Object creation
\newcommand{\new}[3]{\textbf{val }#1 = \textbf{new }#2 ;\; #3}
\newcommand{\anfmapp}[5]{\textbf{val }#1 = {#2.#3(#4)} ;\; #5}
\newcommand{\anfmexe}[5]{\textbf{val }#1 = {#2.#3\ldots\;#4} ;\; #5}
%\renewcommand{\new}[3]{#1 \leftarrow #2 \,\textbf{in}\, #3}
% Field declaration
\newcommand{\Ldecl}[3]{#1 : #2..#3}%{#1 \operatorname{>:} #2 \operatorname{<:} #3}
\newcommand{\ldecl}[2]{#1 : #2}
\newcommand{\mdecl}[3]{#1 : #2 \tfun #3}
% Top and Bottom
\newcommand{\Top}{\top}%{\textbf{Top}}
\newcommand{\Bot}{\bot}%\textbf{Bot}}
% Environment extension
%\newcommand{\envplus}[1]{\uplus \{ #1 \}}
\newcommand{\envplus}[1]{, #1}
% Reduction
\newcommand{\reduction}[2]{#1 \reduces #2 }
\DeclareMathOperator{\klet}{\mathbf{let}} % Used in examples
\DeclareMathOperator{\tlet}{\mathbf{let}} % Used in text


\DeclareMathOperator{\kin}{\mathbf{in}} % Used in examples
\DeclareMathOperator{\tin}{\mathbf{in}} % Used in text

\DeclareMathOperator{\kfix}{\mathbf{fix}}
\DeclareMathOperator{\tfix}{\mathbf{fix}} % Used in text
\DeclareMathOperator{\kif}{\mathbf{if}}
\DeclareMathOperator{\kthen}{\mathbf{then}}
\DeclareMathOperator{\kelse}{\mathbf{else}}
\DeclareMathOperator{\kint}{\mathbf{Int}}
\DeclareMathOperator{\kbool}{\mathbf{Bool}}
\DeclareMathOperator{\ktrue}{\mathbf{true}}
\DeclareMathOperator{\kfalse}{\mathbf{false}}
%\newcommand{\btyp}{B}
\newcommand{\btyp}{\iota}

\newcommand{\calculus}{the calculus\xspace}

% ----- begin macros
\lstdefinelanguage{Scala}%
{morekeywords={abstract,%
  case,catch,char,class,%
  def,else,extends,final,for,%
  if,import,implicit,%
  match,module,%
  new,null,%
  object,override,%
  %package,% commented out for a specific example
  private,protected,public,%
  for,public,return,super,%
  this,throw,trait,try,type,%
  val,var,%
  with,while,%
  yield,%
  macro%
  },%
  sensitive,%
  morecomment=[l]//,%
  morecomment=[s]{/*}{*/},%
  morestring=[b]",%
  morestring=[b]',%
  showstringspaces=false%
}[keywords,comments,strings]%

\lstset{language=Scala,%
  mathescape=true,%
%  columns=[c]fixed,%
%  basewidth={0.5em, 0.40em},%
  aboveskip=1pt,%\smallskipamount,
  belowskip=1pt,%\negsmallskipamount,
  lineskip=-0.2pt,
  basewidth={0.54em, 0.4em},%
%  basicstyle=\ttfamily,%\scriptsize,%
  basicstyle=\footnotesize\ttfamily,
  keywordstyle=\sffamily\bfseries%
%  keywordstyle=\sffamily\bfseries,%
%  xleftmargin=0.5cm
}


\newcommand{\commentstyle}[1]{\slseries{#1}}
\newcommand{\keywordstyle}[1]{\bfseries{#1}}

% Code
\lstnewenvironment{listing}{\lstset{language=Scala}}{}
\lstnewenvironment{listingsmall}{\lstset{language=Scala,basicstyle=\small\ttfamily}}{}
\lstnewenvironment{listingtiny}{\lstset{language=Scala,basicstyle=\scriptsize\ttfamily}}{}
\lstnewenvironment{lstparagraph}{\lstset{language=Scala}\vspace{1.8mm}}{\vspace{1.8mm}}

\newcommand{\scode}[1]{\lstinline[language=Scala,columns=fixed,basicstyle=\ttfamily,keywordstyle=\ttfamily]|#1|}
\newcommand{\jcode}[1]{\lstinline[language=Java,flexiblecolumns=true,basicstyle=\ttfamily]{#1}}

\newcommand{\code}[1]{\scode{#1}}
\newcommand{\sct}[1]{\S \ref{#1}}

% TODOs:
\newif\ifshowTodos
\showTodostrue
%\showTodosfalse % Uncomment this to hide all TODOs.
\ifshowTodos
\newcommand{\todo}[1]{{\color{red} \textbf{[TODO: #1 ]}}}
\else
\newcommand{\todo}[1]{}
\fi
% paper specific commands


\begin{document}

\title{Compile-Time Views: Predictable Type-Directed Partial Evaluation Without Code Duplication}

\maketitle

\begin{abstract}
Staging systems choose a compilation stage in which a term is executed based on: quotation (e.g., MetaOCaml), or types (e.g., LMS). Type based staging systems, require type annotations of all future stage terms, as well as implementing reification and code generation logic for all future stage types. Further, when we use staging at host language compile-time, all terms scheduled to execute at runtime require type annotations and all libraries used at runtime require logic for reification and code generation. Number of annotations is especially noticeable in languages with local type inference as method parameter types must be provided by the programmer.

We introduce a type based staging system for Scala where terms whose types are annotated are executed in the earlier stage of compilation, in our case, at host language compile-time. Annotated types represent merely compile-time views of original types and therefore no reification and code generation logic is necessary. We compare our framework with LMS and show that in majority of programs we require less type annotations while we achieve same performance improvements.

\end{abstract}
% \category{D.3.3}{Programming Languages}{Language Constructs and Features}

\keywords
Partial Evaluation


\section{Introduction}
\label{sct:introduction}

\emph{Partial evaluation}~\cite{jones1993partial} is an optimization technique
that identifies \emph{statically known} program parts and pre-computes them at
compile time. The compile-time computation yields a \emph{residual program} that
does not contain the, previously identified, statically known parts of the
program.  Partial evaluation has been intensively studied and successfully
applied for: removing abstraction overheads in high-level
programs~\cite{carette2005multi,rompf2012lightweight}, domain-specific
languages~\cite{brady2010,jonnalagedda2014staged}, and converting language
interpreters into compilers~\cite{futamura1999partial,lancet,wurthinger2013one}.
Applying partial evaluation in these domains often improves program performance
by several orders of magnitude~\cite{shali2011Hybrid,brady2010}.

Unlike other compiler optimizations partial evaluation is not \emph{safe} as it
 might lead to \emph{code explosion} and might not \emph{terminate}. Due to
 compile-time execution,  computing \code{fold}s and loops over data structures
 of static size can produce arbitrarily  large programs. Furthermore, in a
 Turing-complete language assuring termination is undecidable.

Automatically assuring safety of partial evaluators necessarily leads to
 lack of \emph{predictability}. To illustrate, let us define a function
 \code{dot} for computing a dot-product of two vectors that contain numeric values
 \footnotetext{We use Scala for all code examples in this paper. In order to
 comprehend the paper the reader is required to know the mere basics of the
 language}.

\vspace{1.8mm}
\begin{listing}
  def dot[V:Numeric](v1: Vector[V], v2: Vector[V]): V =
    (v1 zip v2).foldLeft(zero[V]){ case (prod, (cl, cr)) =>
      prod + cl * cr
    }
\end{listing}
\vspace{1.8mm}

When \code{dot} is called with vectors of static size (\eg \code{dot(Vector(2,
 4), Vector(1, 10))}) the abstraction overhead of \code{zip} and \code{foldLeft}
 can be completely removed. However, the partial evaluator must apply extensive
 analysis to conclude that vectors are static in size and that this can be later
 used to unroll the recursion inside \code{foldLeft}. Even if the analysis is
 successful the evaluator must be conservative about unrolling the
 \code{foldLeft}. The vector sizes, and thus the produced code, can unacceptably
 large in a general case. What if we know that vector sizes are relatively small
 and we would like to predictably unroll \code{dot}  into a flat sum of products?


Lack of predictability and danger of code explosion are the reason that
 successful partial evaluators
 \cite{brady2010,taha_multi-stage_1997,rompf2012lightweight,wurthinger2013one,le2004specialization}
  are programmer controlled. We categorize the existing solutions in three categories
 (for further discussion \cf \sct{sct:related-work}):

\begin{itemize}
 \item Programming languages Idris and D provide allow placing the \code{static}
  annotation on function arguments. Since \code{static} is placed on terms, it
  denotes that the \emph{whole term} is static. This restricts the number of programs
  that can be expressed, \eg, we could not express in the signature of \emph{dot} that
  vector parameters are static in size.

 \item Type-directed partial evaluation~\cite{danvy1999type} and
  Lightweight Modular Staging~\cite{rompf2012lightweight} use types to communicate
  the programmer's intent about partial evaluation. By changing the types of parameters
  to be (\eg \code{Vector[Rep[T]]} these approaches can express that parameter vectors
  are statically known.However, they still require existence of two data structures
  (\eg \code{Rep[Vector]} and \code{Vector}. This fosters, costly and hardly maintainable, code duplication.

 \item MetaOCaml~\cite{taha_multi-stage_1997} places terms in, possibly nested,
   quotes. Depth of the term in the quotes denotes the stage of the computation
   where it will be executed. In MetaOCaml we can express the \code{dot} function,
   but we have to modify the code of the \code{dot} function which might not be desirable.

\end{itemize}


% To show how programmers can control partial evaluation we

Ideally, a programmer would with a minimal number of annotations be able to:
 \emph{i)} require that input vectors are of statically known size but polymorphic
  in their elements, \emph{ii)} without modifying the terms require that all operations
  on vector arguments are further partially evaluated, \emph{iii)} allow elements
  of vectors to be generic, and \emph{iv)} reuse the existing implementation of
  the \code{Vector} data structure.

The main idea of this paper is to explicitly capture the user's intent about partial
evaluation in the types. We annotate every type in the language with one of the three values:

\begin{itemize}
 \item $dynamic$ signifies that the value of the type is not known at compile-time. In code $dynamic$ is represented as \code{@d}.
 \item $static$ signifies that the value is known at compile-time. In code $static$ is represented as \code{@s}.
 \item $inline$ requires that the type is statically known and guarantees that operations on the term will be partially evaluated. In code $inline$ is represented as \code{@i}.
\end{itemize}

With our partial evaluator, to require that vectors \code{v1} and \code{v2} are static and
to partially evaluate the function, a programmer would need to make a simple modification of
the \code{dot} signature:
\vspace{1.8mm}
\begin{listing}
  def dot[V: Numeric](v1: Vector[V] @i, v2: Vector[V] @i): V
\end{listing}
\vspace{1.8mm}
This, in effect, requires that only vector arguments (not their elements) are statically known and that all operations will be inlined and further partially evaluated. Residual programs of \code{dot} application in different cases are:

\vspace{1.8mm}
\begin{listing}[mathescape]
  // [el1, el2, el3, el4] are dynamic
  dot(Vector(el1, el2), Vector(el3, el4))
    $\hookrightarrow$ el1 * el3 + el2 * el4

  dot(Vector(2, 4), Vector(1, 10))
    $\hookrightarrow$ 2 * 1 + 4 * 10

  // inline promotes static terms into inline
  dot(Vector(inline(2), inline(4)), Vector(inline(1), inline(10)))
    $\hookrightarrow$ 42
\end{listing}
\vspace{1.8mm}

Contributions:

Evaluation:

Sections:

t\clearpage
\section{The \calculus Calculus}
\label{sct:calculus}

\begin{figure}[H]
\begin{multicols}{2}
\syntaxfig{
  S,\ T,\ U ::=                     & \lindent{\mbox{Types:}}              \\
  \gap \i{S} \ra \j{T}              & \mbox{function type}                 \\
  \gap \{ \seq{x: \i{S}} \}         & \mbox{record type}                   \\
  \gap [X <: \i{S}] \ra \j{T}       & \mbox{universal type}                \\
  \gap Any                          & \mbox{top type}                      \\
  \i{T},\ \j{T},\ \k{T},\ \l{T} ::= & \lindent{\mbox{Binding-Time Types:}} \\
  \gap X                            & \mbox{type identifier}               \\
  \gap \dynamic{T}                  & \mbox{dynamic type}                  \\
  \gap \inline{T}                   & \mbox{inline type}                   \\
  \Gamma ::=                        & \lindent{\mbox{Contexts:}}           \\
  \gap \emptyset                    & \mbox{empty context}                 \\
  \gap \Gamma,\ x: \i{T}            & \mbox{term binding}                  \\
  \gap \Gamma,\ X <: \i{T}          & \mbox{type binding}                  \\
}
\syntaxfig{
  t ::=                             & \lindent{\mbox{Terms:}}              \\
  \gap x,\ y                        & \mbox{identifier}                    \\
  \gap \i{v}                        & \mbox{literal}                       \\
  \gap \dynamic{t}                  & \mbox{dynamic coercion}              \\
  \gap t(t)                         & \mbox{application}                   \\
  \gap t.x                          & \mbox{selection}                     \\
  \gap t [\i{T}]                    & \mbox{type application}              \\
  \i{v} ::=                         & \mbox{Literals:}                     \\
  \gap v                            & \mbox{dynamic literal}               \\
  \gap \inline{v}                   & \mbox{inline literal}                \\
  v ::=                             & \lindent{\mbox{Values:}}             \\
  \gap (x: \i{T}) \ra t             & \mbox{function value}                \\
  \gap \{ \seq{x = t} \}            & \mbox{record value}                  \\
  \gap [X <: \i{T}] \ra t           & \mbox{type abstraction value}        \\
}
\end{multicols}
\caption{Syntax of \calculus}
\end{figure}

We formalize the essence of our partial evaluation system in a minimalistic calculus
based on kernel \fsub \cite{tapl} with lazy records. To accommodate predictable partial evaluation we
introduce binding-time annotations into the type system as types that
represent two kinds of bindings:

\begin{enumerate}
  \item \textbf{Dynamic}. Corresponds to terms that are expected to be evaluted at runtime.
  \item \textbf{Inline}. Corresponds to terms that must be evaluated at compile-time.
\end{enumerate}

To simplify judgments in our formalization we use concise $\i{T}$ syntax to abstract over
binding times of types. Here $i$ signifies the bit of information that says if type is inline
or not and $T$ carries underlying type that is being annotated. So for example in $\inline{Any}$
we get $i = inline$ and $T = Any$.

Similarly we abstract over binding times of literals through $\i{v}$ notation that has analogous
interpretation to the one we use for types.

\subsection{Type composition}

\subsection{Well-formed types}

Even though binding-time information is represented as types, not all
of the possible combinations of types and binding-times is correct.
We restrict types to disallow nesting of more specific binding times
into less specific ones.

\begin{figure}[H]
  \infax[\textsc{W-Any}]
  {\wft{\i{Any}}}

  \infrule[\textsc{W-Abs}]
  {i \leq j \ \ \ \ i \leq k \ \ \ \ \wft{\j{T_1}} \ \ \ \ \wft{\k{T_2}}}
  {\wft{\i{(\j{T_1} \ra \k{T_2})}}}

  \infrule[\textsc{W-TAbs}]
  {i \leq j \ \ \ \ i \leq k \ \ \ \ \wft{[X \mapsto \j{S}] \k{T}}}
  {\wft{\i{([X <: \j{S}] \ra \k{T})}}}

  \infrule[\textsc{W-Rec}]
  {i \leq \seq{j} \ \ \ \ \seq{\wft{\j{T}}}}
  {\wft{\i{\{\seq{x: \j{T}}\}}}}
\caption{Well-formed types $\wft{\i{T}}$.}
\end{figure}

We represent notion of more specific binding-times through
a simple partial order on binding time annotations.

\begin{figure}[H]
  \begin{center}
  $ dynamic \leq dynamic $ \\
  $ inline  \leq dynamic $ \\
  $ inline  \leq inline  $
  \end{center}
\caption{Partial order on binding-time $i \leq j$}
\end{figure}

This restriction allows us to reject programs that have inconsistent binding-time annotations.
For example the following function has incorrectly annotated parameter binding time:
\begin{equation}\nonumber
    (x: \inline{Int}) \ra x + 1
\end{equation}

This is inconsistent because a dynamic function may not have any non-dynamic parameters.
As described in \textsc{W-Abs} functions may only have parameters that are at most as specific
as function binding-time. In our example this doesn't hold as $inline$ is more
specific than $dynamic$.

\subsection{Subtyping}

\calculus integrates binding-time annotation into subtyping relation on regular types by
threading inlining information throughout all of the standard subtyping rules.

\begin{figure}[H]
  \infax[\textsc{S-Top}]
  {\Gamma \ts \i{S} <: \i{Any}}

  \infax[\textsc{S-Refl}]
  {\Gamma \ts \i{S} <: \i{S}}

  \infrule[\textsc{S-Trans}]
  {\Gamma \ts \i{S} <: \j{U} \ \ \ \ \Gamma \ts \j{U} <: \k{T}}
  {\Gamma \ts \i{S} <: \j{U}}

  \infrule[\textsc{S-Arrow}]
  {\Gamma \ts \k{T_1} <: \i{S_1} \ \ \ \ \Gamma \ts \j{S_2} <: \l{T_2}}
  {\Gamma \ts \i{S_1} \ra \j{S_2} <: \k{T_1} \ra \l{T_2}}

  \infrule[\textsc{S-All}]
  {\Gamma,\ X <: \i{U_1} \ts \j{S_2} <: \k{T_2}}
  {\Gamma \ts [X <: \i{U_1}] \ra \j{S_2} <: [X <: \i{U_1}] \ra \k{T_2}}

  \infrule[\textsc{S-Perm}]
  {\{ x_p : i_p S_p\ ^{p \in 1..n}\} $ is permutation of $ \{ y_p: j_p T_p\ ^{p \in 1..n} \}}
  {\Gamma \ts \{ x_p : i_p S_p\ ^{p \in 1..n}\} <: \{ y_p: j_p T_p\ ^{p \in 1..n} \} }

  \infrule[\textsc{S-Depth}]
  {\forall p \in 1..n.\ \Gamma \ts i_p S_p <: j_p T_p}
  {\Gamma \ts \{ x_p: i_p S_p\ ^{p \in 1..n}\} <: \{ x_p: j_p T_p\ ^{p \in 1..n} \}}

  \infax[\textsc{S-Width}]
  {\Gamma \ts \{ x_p: i_p T_p\ ^{p \in 1..n+m} \} <: \{ x_p: i_p T_p\ ^{p \in 1..n} \}}
\caption{Subtyping $\Gamma \ts T_1 <: T_2$.}
\end{figure}

Apart from that we also introduce subtyping on binding-time types.

\begin{figure}[H]
  \infrule[\textsc{S-TVar}]
  {X <: \i{T} \in \Gamma}
  {\Gamma \ts X <: \i{T}}

  \infrule[\textsc{S-Inline}]
  {i = j \ \ \ \ \Gamma \ts S <: T}
  {\Gamma \ts \i{S} <: \j{T}}
\caption{Subtyping of binding-time types $\Gamma \ts \i{T_1} <: \j{T_2}$.}
\end{figure}

Two binding-time types are subtypes if their underlying types are
subtypes and if they have the same binding time.

\subsection{Type polymorphism}

Our system retains traditional type abstraction means inherited from \fsub.
We extend it to accomodate encoding of binding-times into types. This allows
us to specify binding type of the abstracted generic type:
\begin{equation}\nonumber
  [T <: \dynamic{Any}] \ra (x: T) \ra x
\end{equation}

For this particular identity function we need to restrict subset of all admissible
types to only allow $dynamic$ ones. Passing an $inline$ type would not make sense
as the resulting type would have not been well-formed.

\subsection{Typing}

\begin{figure}[H]
  \infrule[\textsc{T-Ident}]
  {x: \i{T} \in \Gamma}
  {\Gamma \ts x: \i{T}}

  \infrule[\textsc{T-Rec}]
  {\Gamma \ts \seq{t}: \seq{\j{T}} \ \ \ \ \wft{\i{\{\seq{x: \j{T}}\}}}}
  {\Gamma \ts \i{\{ \seq{x = t} \}} : \i{\{\seq{x: \j{T}}\}}}

  \infrule[\textsc{T-App}]
  {\Gamma \ts t_1: \i{(\j{T_1} \ra \k{T_2})} \ \ \ \ \Gamma \ts t_2: \j{T_1}}
  {\Gamma \ts t_1 (t_2) : \k{T_2}}

  \infrule[\textsc{T-Sel}]
  {\Gamma \ts t: \i{\{ x = \j{T_1}, \seq{y = \k{T_2}} \}}}
  {\Gamma \ts t.x : \j{T_1}}

  \infrule[\textsc{T-Sub}]
  {\Gamma \ts t: \i{S} \ \ \ \ \Gamma \ts \i{S} <: \j{T}}
  {\Gamma \ts t: \j{T}}

  \infrule[\textsc{T-Abs}]
  {\Gamma,\ x: \j{T_1} \ts t: \k{T_2} \ \ \ \ \wft{\i{(\j{T_1} \ra \k{T_2})}}}
  {\Gamma \ts \i{((x: \j{T_1}) \ra t)} : \i{(\j{T_1} \ra \k{T_2})}}

  \infrule[\textsc{T-TAbs}]
  {\Gamma,\ X <: \j{T_0} \ts t_2: \k{T_2} \ \ \ \ \wft{\i{([X <: \j{T_1}] \ra \k{T_2})}}}
  {\Gamma \ts \i([X <: \j{T_1}] \ra t_2): \i{([X <: T_1] \ra \k{T_2})}}

  \infrule[\textsc{T-TApp}]
  {\Gamma \ts t: \i{([X <: \j{T_1}] \ra \k{T_2})}  \ \ \ \ \Gamma \ts \l{T} <: \j{T_1}}
  {\Gamma \ts t[\l{T}] : [X \mapsto \l{T}] \k{T_2}}

  \infrule[\textsc{T-Dynamic}]
  {\Gamma \ts t: \inline{T}}
  {\Gamma \ts \dynamic{t} : \dynamic{T}}
\caption{Typing $\Gamma \ts t: \i{T}$.}
\end{figure}

Similarly to the changes made to the subtyping relation we thread binding-time information
throughout typing relation. Apart from that we also ensure that all literals produced by the
user have well-formed types.

\subsection{Partial Evaluation}

\begin{figure}[H]
  \infrule[\textsc{PE-Abs}]
  {t \pe t'}
  {(x: \i{T}) \ra t \pe (x: T) \ra t'}

  \infrule[\textsc{PE-Rec}]
  {\seq{t} \pe \seq{t'}}
  {\{\seq{x = t}\} \pe \{\seq{x = t'}\}}

  \infrule[\textsc{PE-TAbs}]
  {t \pe t'}
  {[X <: \i{T}] \ra t \pe [X <: \i{T}] \ra t'}

  \infrule[\textsc{PE-App}]
  {t_1 \pe t_1' \ \ \ \ t_1 \ne \inline{t_3} \ \ \ \ t_2 \pe t_2'}
  {t_1(t_2) \pe t_1'(t_2')}

  \infrule[\textsc{PE-Sel}]
  {t \pe t' \ \ \ \ t' \ne \inline{t_3}}
  {t.x \pe t'.x}

  \infrule[\textsc{PE-TApp}]
  {t_1 \pe t_1' \ \ \ \ t_1' \ne \inline{t_3}}
  {t_1[\i{T_1}] \pe t_1'[\i{T_1}]}

  \infrule[\textsc{PE-InlineApp}]
  {t_1 \pe \inline{(x: \i{T}) \ra t} \ \ \ \ t_2 \pe t_2' \ \ \ \ [x \mapsto t_2'] t \pe t'}
  {t_1(t_2) \pe t'}

  \infrule[\textsc{PE-InlineSel}]
  {t \pe \inline{\{x = t_x,\ \seq{y = t_y}\}} \ \ \ \ t_x \pe t_x'}
  {t.x \pe t_x'}

  \infrule[\textsc{PE-InlineTApp}]
  {t_1 \pe \inline [X <: \i{T_2}] \ra t_2 \ \ \ \ [X \mapsto \i{T_1}] t_2 \pe t_2'}
  {t_1[\i{T_1}] \pe t_2'}

  \infax[\textsc{PE-InlineValue}]
  {\inline{v} \pe \inline{v}}

  \infrule[\textsc{PE-Dynamic}]
  {t \pe \inline{t'} \ \ \ \ t' \pe t''}
  {\dynamic{t} \pe t''}
\caption{Partial evaluation $t \pe t'$}
\label{fig:partial-evaluation}
\end{figure}

\subsection{Evaluation}

Once partial evaluation is complete we strip all binding-time annotations on types
and convert inline terms into corresponding dynamic ones. After that we can use
standard \fsub evaluation rules augmented with lazy records semantics (\textsc{E-Sel}).

\begin{figure}[H]
  \infax[\textsc{E-Value}]
  {v \e v}

  \infrule[\textsc{E-App}]
  {t_1 \e (x: T) \ra t \ \ \ \ t_2 \e v \ \ \ \ [x \mapsto v] t \e v'}
  {t_1(t_2) \e v'}

  \infrule[\textsc{E-TApp}]
  {t_1 \e [X <: T_2] \ra t_2 \ \ \ \ [X \mapsto T_1] t_2 \e v}
  {t_1[T_1] \e v}

  \infrule[\textsc{E-Sel}]
  {t \e \{ x = t_x,\ \seq{y = t_y}\} \ \ \ \ t_x \e v}
  {t.x \e v}
\caption{Evaluation $t \e v$}
\end{figure}

\subsection{Conjectures}

\begin{enumerate}
  \item Progress and preservation of partial evaluation.
  \item Progress and preservation of evaluation.
\end{enumerate}


\section{Translating Object Oriented Features Into \calculus}
\label{sct:scala-translation}
\usetikzlibrary{arrows, decorations.markings}
% for double arrows a la chef
% adapt line thickness and line width, if needed
\tikzstyle{vecArrow} = [thick, decoration={markings,mark=at position
   1 with {\arrow[semithick]{open triangle 60}}},
   double distance=1.4pt, shorten >= 5.5pt,
   preaction = {decorate},
   postaction = {draw,line width=1.4pt, white,shorten >= 4.5pt}]
\tikzstyle{innerWhite} = [semithick, white,line width=1.4pt, shorten >= 4.5pt]

\begin{figure}
\center
\begin{tikzpicture}[thick]
  \node[draw,rectangle] (a) {Desugaring to \calculus};
  \node[draw,rectangle,below of=a, node distance = 1.0cm] (b) {Type Checking \calculus};
  \node[draw,rectangle,right of=b, node distance = 3.5cm] (c) {Type Erasure};
  \node[draw,rectangle,right of=a, node distance = 3.5cm] (d) {Partial Evaluation};
  \node[draw,rectangle,right of=d, node distance = 3.5cm] (e) {Inline Erasure};
  \node[draw,rectangle,right of=c, node distance = 3.5cm] (f) {Runtime Evaluation};

  % 1st pass: draw arrows
  \draw[vecArrow] (a) to (b);
  \draw[vecArrow] (b) to (c);
  \draw[vecArrow] (c) to (d);
  \draw[vecArrow] (d) to (e);
  \draw[vecArrow] (e) to (f);

  % Note: If you have no branches, the 2nd pass is not needed
\end{tikzpicture}
\caption{Compilation pipeline.}
\label{fig:phases}
\end{figure}

% Description of the chapter

The core calculus \sct{sct:calculus} captures the essence of user-controlled
 predictable partial-evaluation. In practice, though, it requires an inconveniently large number
 of \code{inline} calls. Moreover, the calculus does not provide a way to define
 data structures that would correspond to \emph{classes} in modern multi-paradigm languages.
 In this section we formalize convenient implicit conversions for the calculus \sct{sct:conversions}, a scheme for
 translating classes into the calculus how to promote constructs classes and \emph{methods} into their
 partially-evaluated versions \sct{sct:promotion}.

% Restricted Language

The core rules of the calculus do not support effect-full computations and each
 \code{inline} term is trivially converted to a dynamic term after erasure.
 In case of languages that do support mutable state and side-effects this needs to
 be treated specially. For simplicity, we omit side-effects from our discussion and
 assume that all partially evaluated code is side-effect free and that each
 \code{inline} term can be converted to dynamic code.

\subsection{Desugaring Object Oriented Constructs to \calculus}
\label{sct:desugaring}

\begin{figure}
\begin{alignat*}{2}
   & [\![ \klet\ x: T_x = t_x\ \kin \ t ]\!] = ((x: T_x) \ra t)(t_x)\\
   & [\![ \klet\ type\ T_1 = T_2\ \kin\ t ]\!] =  ([T_1 <: T_2] \ra t)[T_2] \\
   & [\![ \klet\ class\ C[A](x: T_x) \{ def\ f[B](y: T_y) = t_f \}\ \kin\ t ]\!]  =  \\
   & \quad   \klet\ type\  C = [A] \ra \{ x: T_x, f: [B] \ra T_y \ra T_f \}\ \kin  \\
   & \quad \quad \klet\ C: [A] \ra (x: T_x) \ra C[A]  =  \\
   & \quad \quad \quad [A] \ra (x: T_x) \ra \{ x = x, f = [B] \ra (y: T_y) \ra t_f \}\ \kin\ t
\end{alignat*}
\caption{Desugaring of classes into \calculus.}
\label{fig:desugaring-classes}
\end{figure}

\subsection{Compile-Time View of Terms}
\label{sct:compile-views}
\begin{figure}
\begin{multicols}{2}[]

  \infrule[\textsc{CT-TVar}]
    {\Pi \ts T \in \Pi}
    {\Pi \ts \pe{iT}{iT}}

  \infrule[\textsc{CT-T-Var}]
    {\Pi \ts T \not\in \Pi}
    {\Pi \ts \pe{iT}{\inline{T}}}

\end{multicols}
\vspace{4pt}

  \infrule[\textsc{CT-Rec}]
    {\seq{\Pi \ts \pe{t}{t'}}}
    {\Pi \ts \pe{\i\{ \seq{x = t} \}}{\inline{\{ \seq{x = t'} \}}}}

  \infrule[\textsc{CT-T-Rec}]
    {\seq{\Pi \ts \pe{\i{T}}{\j{T}}}}
    {\Pi \ts \pe{\{\seq{x : \i{T}}\}}{\inline{\{ \seq{x : \j{T}}}\}}}

  \infrule[\textsc{CT-T-Arrow}]
    {\Pi \ts \pe{\i{T}}{\j{T}} \ \ \ \  \Pi \ts \pe{\k{S}}{\l{S}} }
    {\Pi \ts \pe{\i{T} \Rightarrow \k{S}}{\j{T} \Rightarrow \l{S}}}

  \infrule[\textsc{CT-T-Univ}]
    {\Pi \ts \pe{\j{T}}{\k{T}}}
    {\Pi \ts \pe{[X <: \i{S}] \ra \j{T}}{[X <: \i{S}] \ra \k{T}}}

  \infrule[\textsc{CT-Func}]
    {\Pi \ts \pe{t}{t'} \ \ \ \ \Pi \ts \pe{iT}{jT}}
    {\Pi \ts \pe{\i(x: iT) \ra t}{\inline{(x: jT) \ra t'}}}

  \infrule[\textsc{CT-TAbs}]
    {\Pi,\ X \ts \pe{t}{t'}}
    {\Pi \ts \pe{\i([X <: \j{T_1}] \ra t)}{\inline{([X <: \j{T_1}] \ra t'})}}

  \infrule[\textsc{CT-TApp}]
    {\Pi \ts \pe{t}{t'} \ \ \ \ \Pi \ts \pe{\i{T}}{\j{T}}}
    {\Pi \ts \pe{t[\i{T}]}{t'[\j{T}]}}

\caption{Translation of a regular class to a compile-time version.}
\label{fig:partial-evaluation}
\end{figure}

\section{Case Studies}
\label{sct:case-studies}

In this section we present selected use-cases for compile-time views that at the
same  time demonstrate step-by-step the mechanics behind \tool and the
interesting applications.  We start by inlining a simple function with staging
(\sct{sct:inlining}), then do the canonical staging  example of the power function
(\sct{sct:recursion}), then we demonstrate how variable argument functions can
be  desugared into the core functionality (\sct{sct:varargs}). Finally, we
demonstrate how the abstraction overhead of the \code{dot} function and all
associated type-class related abstraction an be removed (\sct{sct:dot-product}).
For formal partial evaluation rules refer \cf \sct{fig:partial-evaluation}.

\subsection{Inlining Expressed Through Staging}
\label{sct:inlining}

Function inlining can be expressed as staged computation~\cite{monnier2003inlining}.
 Inlining is achieved when a statically known function body is applied with symbolic
 arguments. In \tool we use the \code{ct} annotation on functions and methods to achieve inlining:\begin{lstparagraph}
@ct def zero[T](implicit num: Numeric[T]) = num.zero

zero[Double]
  $\hookrightarrow$ num.zero
\end{lstparagraph}


\subsection{Recursion}
\label{sct:recursion}

The canonical example in staging literature is partial evaluation of the power function
 where exponent is an integer:
\begin{lstparagraph}
def pow(base: Double, exp: Int): Double =
  if (exp == 0) 1 else base * pow(base, exp - 1)
\end{lstparagraph} When the exponent (\code{exp}) is statically known this function can be partially
evaluated into \code{exp} multiplications of the \code{base} argument, significantly
improving performance~\cite{calcagno2003implementing}.

With compile-time views making\code{pow} partially evaluated requires adding two annotations:

\begin{lstparagraph}
def pow(base: Double, exp: Int@ct): Double =
  if (exp == 0) 1 else base * pow(base, exp - 1)
\end{lstparagraph}

% TODO cite infinite recursion
To satisfy cross-stage persistence (\sct{sct:wf-ctv}) the \code{pow} must be \code{@ct}.
However, to reduce the number of required annotations we implicitly add the \code{ct} annotation
when at least one parameter type or the result type is marked as \code{ct}. In the example
 the \code{ct} annotation on \code{exp} requires that the function must be called with
 a compile-time view of \code{Int}. \tool ensures that the definiton of the \code{pow} function
 does not cause infinite recursion at compile-time~\cite{} by invoking the power function
 only when the value of the \code{ct} arguments is known.

 The application of the function \code{pow} with a constant
 exponent will produce:

\begin{lstparagraph}
pow(base, 4)
  $\hookrightarrow$ base * base * base * base * 1
\end{lstparagraph}

Constant 4 is promoted to \code{ct} by the implicit conversions (\sct{sct:implicits}).

\subsection{Variable Argument Functions}
\label{sct:varargs}

% Variable argument functions
Variable argument functions appear in widely used languages like Java, C\#, and Scala.
 Such arguments are typically passed in the function body inside of the data structure
 (\eg \code{Seq[T]} in Scala). When applied with variable arguments the size of the
 data-structure is statically known and all operations on them can be partially
 evaluated. However, sometimes, the function is called with arguments of dynamic size.
 For example, function \code{min} that accepts multiple integers\begin{lstparagraph}
def min(vs: Int*): Int = vs.tail.foldLeft(vs.head) {
  (min, el) => if (el < min) el else min
}
\end{lstparagraph}can be called either with statically known arguments
 (\eg, \code{min(1,2)}) or with dynamic arguments:\begin{lstparagraph}
val values: Seq[Int] = ... // dynamic value
min(values: _*)
\end{lstparagraph}

Ideally, we would be able to achieve partial evaluation if the arguments are of statically
known size and avoid partial evaluation in case of dynamic arguments. To this end we translate
the method \code{min} into a partially evaluated version and a dynamic version. The call to these
methods is dispatched, at compile-time, by the \code{min} method which checks if
arguments are statically known. Desugaring of \code{min} is shown in \figref{fig:min}.

\begin{figure}
\begin{listing}
def min(vs: Int*): Int = macro
  if (isVarargs(vs)) q"min_CT(vs)"
  else q"min_D(vs)"

def min_CT(vs: Seq[Int]@ct): Int =
  vs.tail.foldLeft(vs.head) { (min, el) =>
    if (el < min) el else min
  }

def min_D(vs: Seq[Int]): Int =
  vs.tail.foldLeft(vs.head) {
    (min, el) => if (el < min) el else min
  }
\end{listing}
\caption{Function \code{min} is desugared into a \code{min} macro that based on the
binding time of the arguments dispatches to the partially evaluated version (\code{min_CT})
for statically known varargs or to the original min function for dynamic arguments \code{min_D}.}
\label{fig:min}
\end{figure}

\subsection{Removing Abstraction Overhead of Type-Classes}
\label{sct:type-classes-removal}

Type-classes are omnipresent in everyday programming as they provide allow abstraction over
 generic parameters (\eg, \code{Numeric} abstracts over numeric values). Unfortunately,
 type-classes introduce \emph{dynamic dispatch} on every call~\cite{rompf_optimizing_2013} and are,
 thus, impose a performance penalty. Type-classes are in most of the cases statically known. Here
 we show how with \tool we can remove all abstraction overheads of type classes.

In Scala, type classes are implemented with objects and implicit parameters~\cite{oliveira_type_2010}.
In \figref{fig:numeric}, we define a \code{trait Numeric} serves as an interface for
all numeric types. Then we define a concrete implementation of \code{Numeric} for
type \code{Double} (\code{DoubleNumeric}). The \code{DoubleNumeric} is than passed
as an implicit argument \code{dnum} to all methods that use it (\eg, \code{zero}).

\begin{figure}
\begin{listing}
object Numeric {
  implicit def dnum: Numeric[Double]@ct =
    ct(DoubleNumeric)
  def zero[T](implicit num: Numeric[T]@ct): T =
    num.zero
}

trait Numeric[T] {
  def plus(x: T, y: T): T
  def times(x: T, y: T): T
  def zero: T
}

object DoubleNumeric extends Numeric[Double] {
  def plus(x: Double, y: Double): Double = x + y
  def times(x: Double, y: Double): Double = x * y
  def zero: Double = 0.0
}
\end{listing}
\caption{\label{fig:numeric} Removing abstraction overheads of type classes.}
\end{figure}

When \code{zero} is applied first the implicit argument (\code{dnum}) gets
inlined due to the \code{ct} annotation of the return type, then the function \code{zero} gets
inlined. Since \code{dnum} returns a compile-time view of \code{DoubleNumerc}
the method \code{zero} on \code{dnum} is evaluated at compile time. The constant \code{0.0} is
promoted to \code{ct} since \code{DoubleNumeric} is a compile time view. Finally the \code{ct(0.0)} result
is coerced to a dynamic value by the signature of \code{Numeric.zero}. The
compile-time execution is shown in the following snippet

\begin{lstparagraph}
Numeric.zero[Double]
  $\hookrightarrow$ Numeric.zero[Double](DoubleNumeric)
  $\hookrightarrow$ ct(DoubleNumeric).zero
  $\hookrightarrow$ (ct(0.0): Double)
  $\hookrightarrow$ 0.0
\end{lstparagraph}

\subsection{Inner Product of Vectors}
\label{sct:dot-product}

Here we demonstrate how the introductory example (\sct{sct:introduction}) is
partially evaluated through staging. We start with the desugared \code{dot}
function~(\ie, all implicit operations are shown):

\begin{lstparagraph}
 def dot[V](v1: Vector[V]@ct, v2: Vector[V]@ct)
  (implicit num: Numeric[V]@ct): V =
  (v1 zip v2).foldLeft(zero[V](num)) {
    case (prod, (cl, cr)) => prod + cl * cr
  }
\end{lstparagraph}

Function \code{dot} is generic in the type of vector elements. This will reflect
upon the staging annotations as well (\code{ct} and \code{static}). When we apply the
\code{dot} function with static arguments we will get the vector with static elements back:

\begin{lstparagraph}
dot[Double@static](
  ct(Vector)(2.0, 4.0), ct(Vector)(1.0, 10.0))(
  Numeric.dnum)
$\hookrightarrow$
  (ct(Vector)(2.0, 4.0) zip ct(Vector)(1.0, 10.0))
    .foldLeft(ct(0.0)) {
      case (prod, (cl, cr)) => prod + cl * cr
    }
$\hookrightarrow$ (2.0 * 1.0 + 4.0 * 10.0): Double@static
\end{lstparagraph}

When \code{dot} is evaluated with the \code{ct} elements the last step will further
execute to a single compile-time value that can further be used in compile-time computations:
\begin{lstparagraph}
dot[Double@ct](
  ct(Vector)(ct(2.0), ct(4.0)),
  ct(Vector)(ct(1.0), ct(10.0)))(Numeric.dnum)
$\hookrightarrow$ ct(2.0) * ct(1.0) + ct(4.0) * ct(10.0)
$\hookrightarrow$ 42.0: Double@ct
\end{lstparagraph}

\section{Compile-Time Views for Scala}
\label{sct:interface}

We have implemented \tool, a staging extension for Scala based on compile-time views.
 \tool is a compiler plugin that executes in a phase after the
 Scala type checker. The plugin starts with pre-typed Scala programs and uses
 type annotations~\cite{odersky_1996_putting} to track and verify information about the biding-time
 of terms. Currently, it supports only two stages of compilation: host language compile-time
 (types annotated with \code{@ct}) and host language run-time (unannotated code).

To the user, \tool exposes a minimal interface (\figref{fig:interface}) with
annotations \code{inline} and \code{ct}, and functions \code{inline} and \code{ct}.

\begin{figure}
\begin{listing}
package object scalact {

  final class ct extends StaticAnnotation
  final class inline extends StaticAnnotation

  @compileTimeOnly def ct[T](body: => T): T = ???
  @compileTimeOnly def inline[T](body: => T): T = ???

}
\end{listing}
\label{fig:interface}
\caption{Interface of the \tool.}
\end{figure}

\smartparagraph{Annotation \code{ct}} is used at the type level~(\eg,
\code{Int@ct}) and denotes a compile-time view of a type. The
annotation is integrated in the Scala's type system  and, therefore, can be
arbitrarily nested in different variants of types. Table  \ref{tbl:ct-type}
shows how the \code{@ct} annotation can be placed on types and how it, due to
the translation to the compile-time views (\figref{ct-translation}), changes
method signatures on annotated types.

\begin{table*}[t]
\caption{Types and corresponding method signatures after the translation to the compile-time view.}
\label{tbl:ct-type}
\centering
\begin{tabularx}{\linewidth}{ X X X X }
\toprule

  Annotated Type              & \ &  Type's Method Signatures                          &  \\
  \code{Int@ct}               & \ &  \code{+(rhs: Int@ct): Int@ct}                     &  \\
  \code{Vector[Int]@ct}       & \ &  \code{map[U](f: (Int => U)@ct): Vector[U]@ct}     &  \\
                              & \ &  \code{length: Int@ct}                             &  \\
  \code{Vector[Int@ct]@ct}    & \ &  \code{map[U](f: (Int@ct => U)@ct): Vector[U]@ct}  &  \\
                              & \ &  \code{length: Int@ct}                             &  \\
  \code{Map[Int@ct, Int]@ct}  & \ &  \code{get(key: Int@ct): Option[Int]@ct}           &  \\

\bottomrule
\end{tabularx}
\end{table*}

 In \tabref{tbl:ct-type}, \code{Int@ct} is a non-polymorphic type and therefore
 according to the translation to the compile-time view (\ref{fig:ct-translation})
 parameters of all methods will also be compile-time views. On the other hand,
 \code{Vector[Int]@ct} will have parameters of all methods transformed except
 the generic ones. In effect, this, makes higher order combinators of \code{Vector}
 operate on dynamic values, thus, function \code{f} passed to \code{map} accepts
 the dynamic value as input. Type \code{Vector[Int@ct]@ct} has all parts executed
 at compile-time. The return type of the function \code{map} can still be
 both dynamic and a compile-time view: due to the type parameter \code{U}.

\smartparagraph{Annotation inline} can be used only at the term level on statically
 known methods and functions. It denotes that the method/function will be inlined during
 compilation time. In other words, \code{inline} is marking that the function application
 is a compile-time computation and that application should be removed by partial evaluation.
 This is not the first time that inlining is achieved through partial
 evaluation~\cite{monnier2003inlining}.

 Internally \code{inline} can be expressed in terms of the \code{ct} annotation. A method\begin{lstparagraph}
@inline def dot[V: Numeric]
  (v1: Vector[V], v2: Vector[V]): V
\end{lstparagraph}
  will have an internal method type\begin{lstparagraph}
((v1: Vector[V], v2: Vector[V]) => V)@ct
\end{lstparagraph} that can not be written by the users. We choose the name
 \code{inline} to be consistent with the existing Scala \code{inline} annotation.

% TODO This is all total crap without having @inline at the term level (classes,
% objects, traits, methods.
% Motivate: data structure that whenewer instantiated is already ct
% Example: complex number
% inline on classes and their methods
% inline on traits
% inline and inheritance
% TODO traits (only if we mus)
% TODO discuss automatic conversions
% TODO discuss more usability things.

% TODO: Well formedness of types goes here as well
% TODO: Reducing the number of annotations (implicit conversions go here)


\smartparagraph{Functions \code{ct} and \code{inline}} are used at the term level
 for promoting Scala objects and methods/functions into their compile-time views. Without the
 the \code{ct} and \code{inline} we would not be able to instantiate compile-time views of types.
 \tabref{tbl:ct-type} shows how different types of terms are promoted to their
 compile-time views.

\begin{table*}[t]
\caption{Promotion of terms to their compile-time views.}
\label{tbl:ct-type}
\centering
\begin{tabularx}{\linewidth}{ X X }
\toprule

  Promoted Term        \quad \quad \quad & Term's Promoted Type             \\
  \code{ct(Vector)(1, 2, 3)            } & \code{: Vector[Int]@ct        }  \\
  \code{ct(Vector)(ct(1), ct(2), ct(3))} & \code{: Vector[Int@ct]@ct     }  \\
  \code{new (Cons@ct)(1, Nil)          } & \code{: Cons[Int]@ct          }  \\
  \code{new (Cons@ct)(ct(1), ct(Nil))  } & \code{: Cons[Int@ct]@ct       }  \\
  \code{ct((x: Int) => x)              } & \code{: (Int@ct => Int@ct)@ct }  \\
  \code{inline((x: Int) => x)          } & \code{: (Int => Int)@ct       }  \\

\bottomrule
\end{tabularx}
\end{table*}

Function \code{ct} can be applied to objects (\eg, \code{Vector}) to provide a compile-time
 view over their methods. When those objects have generic parameters, \code{ct} be used
 to promote the arguments, and thus, the result types of these functions. When applied,
 on functions \code{ct} promotes the compile-time view as well as its arguments
 and the return type.

Function \code{inline} can be applied only on functions/methods to promote only the function
 (and not the arguments) to their compile time views. This function can be seen as a shallow
 version of \code{ct} that makes only the outer type a compile-time view.

\subsection{Well-Formedness of Compile-Time Views}
\label{sct:wf-ctv}

% Nice description of csp and pointer to the right paper.
Compile-time views

% Examples on types
CTVs

% Examples on functions
Examples on functions.

% Examples on classes
Examples on classes (same limitations from scala).

\subsection{Convenient Implicit Conversions}
\label{sct:wf-ctv}

% Examples


\section{Evaluation}
\label{sct:evaluation}

\subsection{Reduction of Code Duplication}

% Explanation of the problem with type-directed partial evaluation.

\subsection{Performance Comparison}
\begin{table}[h]
\caption{Performance comparison with LMS and hand optimized code.}
\label{tbl:numbers}
\centering
\begin{tabularx}{\linewidth}{ X X X X}
\toprule

  Benchmark                  & Hand Optimized                                    &  LMS      &    Scala Inline            \\
  \code{pow}                 &                                                   &           &                            \\
  \code{min}                 &                                                   &           &                            \\
  \code{dot}                 &                                                   &           &                            \\
  \code{fft}                 &                                                   &           &                            \\

\bottomrule
\end{tabularx}
\end{table}

\section{Related Work}
\label{sct:related-work}

% Staging MetaML

% LMS Type Directed partial evaluation

 Programming languages Idris~\cite{brady2010scrapping} and D~\cite{dlang} try to solve this problem by allowing
  the \code{static} annotation on function arguments. Annotation \code{static} denotes
  that the term is statically known and that all operations on that term should
  be executed at compile-time. However, since \code{static} is placed on terms rather
  then types, it can mark only \emph{whole terms} as static. This restricts the number
  of programs that can be expressed, \eg, we could not express that vectors in the
  signature of \emph{dot} are static only in size. Finally, information about \code{static}
  terms can not be deterministically propagated through return types of functions so \code{static}
  in Idris and D is a partial evaluation construct.

% Specialization Scenarios
% MacroML
% Hybrid Partial Evaluation

\section{Conclusion}
\label{sct:conclusion}

\tool is only the first step towards a staging system for Scala that is fully integrated in the language. For
 complete integration it would be necessery to allow usage of \code{ct} on trait and class definitions
 as well as abstract types.

In \tool annotation of a term determins the stage where it is computed. When \tool is used only for partial evaluation
 this can lead to unnecessary code duplication as users need to provide two versions of the same
 code (one staged and one unstaged). For a practical system it is necessary to extend \tool with
 annotations which denote that terms should be partially evaluated if they are statically known.
 We have achieved this for variable argument funcitons in \sct{sct:varargs} with an ad-hoc mechanism,
 however, we seek for a principled solution.


\bibliographystyle{plain}
\bibliography{vjovanov-lib}

\end{document}
