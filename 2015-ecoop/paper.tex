
\let\oldvec\vec% Store \vec in \oldvec
\documentclass{llncs}
\let\vec\oldvec% Restore \vec from \oldvec
\usepackage{llncsdoc}

% % Build
% \usepackage{subfiles}

% % Math
 \usepackage{amsmath}
 \usepackage{amssymb}
 \usepackage{math}
 \usepackage{mathtools}

% % Graphics
 \usepackage{graphicx}

% % Text stuff
 \usepackage{listings}
 \usepackage[T1]{fontenc}
 \usepackage[utf8]{inputenc}
 \usepackage{url}
 \usepackage[usenames]{color}

% % PL Formulas
 \usepackage{latexsym}
 \usepackage{bcprules}
 \usepackage{prooftree}


% % Layout
 \usepackage{xspace}% space in macros

% % Makes tables look beautiful
 \usepackage{booktabs}
 \usepackage{tabularx}

 \input{macros}

% ----- begin macros
\lstdefinelanguage{Scala}%
{morekeywords={abstract,%
  case,catch,char,class,%
  def,else,extends,final,for,%
  if,import,implicit,%
  match,module,%
  new,null,%
  object,override,%
  %package,% commented out for a specific example
  private,protected,public,%
  for,public,return,super,%
  this,throw,trait,try,type,%
  val,var,%
  with,while,%
  yield,%
  macro%
  },%
  sensitive,%
  morecomment=[l]//,%
  morecomment=[s]{/*}{*/},%
  morestring=[b]",%
  morestring=[b]',%
  showstringspaces=false%
}[keywords,comments,strings]%

\lstset{language=Scala,%
  mathescape=true,%
%  columns=[c]fixed,%
%  basewidth={0.5em, 0.40em},%
  aboveskip=1pt,%\smallskipamount,
  belowskip=1pt,%\negsmallskipamount,
  lineskip=-0.2pt,
  basewidth={0.54em, 0.4em},%
%  basicstyle=\ttfamily,%\scriptsize,%
  basicstyle=\footnotesize\ttfamily,
  keywordstyle=\sffamily\bfseries%
%  keywordstyle=\sffamily\bfseries,%
%  xleftmargin=0.5cm
}


\newcommand{\commentstyle}[1]{\slseries{#1}}
\newcommand{\keywordstyle}[1]{\bfseries{#1}}

% Code
\lstnewenvironment{listing}{\lstset{language=Scala}}{}
\lstnewenvironment{listingsmall}{\lstset{language=Scala,basicstyle=\small\ttfamily}}{}
\lstnewenvironment{listingtiny}{\lstset{language=Scala,basicstyle=\scriptsize\ttfamily}}{}

\newcommand{\scode}[1]{\lstinline[language=Scala,columns=fixed,basicstyle=\ttfamily,keywordstyle=\ttfamily]|#1|}
\newcommand{\jcode}[1]{\lstinline[language=Java,flexiblecolumns=true,basicstyle=\ttfamily]{#1}}

\newcommand{\code}[1]{\scode{#1}}
\newcommand{\sct}[1]{(\S \ref{#1})}

% TODOs:
\newif\ifshowTodos
\showTodostrue
%\showTodosfalse % Uncomment this to hide all TODOs.
\ifshowTodos
\newcommand{\todo}[1]{{\color{red} \textbf{[TODO: #1 ]}}}
\else
\newcommand{\todo}[1]{}
\fi
% paper specific commands


\begin{document}

\title{Scala Inline}

\maketitle

\begin{abstract}
\cite{scala_macros}
\end{abstract}
% \category{D.3.3}{Programming Languages}{Language Constructs and Features}

\keywords
Partial Evaluation, Macros

\section{Introduction}
\label{sct:introduction}

\emph{Partial evaluation}~\cite{jones1993partial} is an optimization technique
that identifies \emph{statically known} program parts and pre-computes them at
compile time. Partial evaluation has been intensively studied \cite{}, and
successfully applied:  for removing abstraction overheads in high-level
programs~\cite{carette2005multi,rompf2012lightweight}, for domain-specific
languages~\cite{brady2010,jonnalagedda2014staged}, and for converting language
interpreters into compilers~\cite{futamura1999partial,lancet,wurthinger2013one}.
Applying partial evaluation in these domains often improves program performance
by several orders of magnitude~\cite{shali2011Hybrid,brady2010}.


To achieve \emph{predictable} and \emph{safe} partial evaluation, however, a
partial evaluator must be controlled by the
programmer~\cite{brady2010,le2004specialization}. Unlike other compiler
optimizations, due to compile-time execution, partial evaluation might not
\emph{terminate}. Furthermore, \emph{code explosion} is possible as the final
program can be arbitrarily large due to compile-time execution. Lack of
predictability and danger of code explosion are the reason that successful
partial evaluators \cite{brady2010,taha_multi-stage_1997,rompf2012lightweight,wurthinger2013one}
are programmer controlled.

To show how programmers can control partial evaluation we define a function
\code{dot}  for computing a dot-product of two vectors that contain numeric
values\footnotetext{We use Scala for all examples in this paper. In order to
comprehend the paper the reader is required to know the mere basics of the
language}.
\vspace{1.8mm}
\begin{listing}
  def dot[V:Numeric](v1: Vector[V], v2: Vector[V]): V =
    (v1 zip v2).foldLeft(zero[V]){ case (prod, (cl, cr)) =>
      prod + cl * cr
    }
\end{listing}
\vspace{1.8mm}

To remove the abstraction overhead of \code{zip} and \code{foldLeft} the partial evaluator
must apply extensive analysis conclude that vectors are static in size and
that this can be later used to unroll the \code{foldLeft}. Even if the analysis
is successful the evaluator must be conservative about unrolling the \code{foldLeft}
as vector sizes, and thus the produced code, can be very large. What if we know
vector sizes and we want to predictably unroll \code{dot} into a flat sum of
products?

Ideally the programmer would, with the minimal number of annotations, be able to: \emph{i)} require
that \code{v1} \code{v2} vectors are of statically known size, \emph{ii)} require that all operations on
vector arguments should be further partially evaluated, \emph{iii)} allow elements of
vectors to be generic, and \emph{iv)} do not require the programmer to re-implement
the whole \code{Vector} data structure to achieve partial evaluation.

The main idea of this paper is to explicitly capture the user intent about partial
evaluation in the types. We annotate every type in the language with one of the three values:
\begin{itemize}
 \item $dynamic$ signifies that the value of the type is not known at compile-time. In code $dynamic$ is represented as \code{@d}.
 \item $static$ signifies that the value is known at compile-time. In code $static$ is represented as \code{@s}.
 \item $inline$ requires that the type is statically known and guarantees that operations on the term will be partially evaluated. In code $inline$ is represented as \code{@i}.
\end{itemize}

With our partial evaluator, to require that vectors \code{v1} and \code{v2} are static and
to partially evaluate the function the programmer would need to make a simple modification of
the \code{dot} signature:
\vspace{1.8mm}
\begin{listing}
  def dot[V: Numeric](v1: Vector[V] @i, v2: Vector[V] @i): V
\end{listing}
\vspace{1.8mm}
This, in effect, requires that only vector arguments (not their elements) are statically known and that all operations will be inlined and further partially evaluated. Residual programs of \code{dot} application in different cases are:

\vspace{1.8mm}
\begin{listing}[mathescape]
  // [el1, el2, el3, el4] are dynamic
  dot(Vector(el1, el2), Vector(el3, el4))
    $\hookrightarrow$ el1 * el3 + el2 * el4

  dot(Vector(2, 4), Vector(1, 10))
    $\hookrightarrow$ 2 * 1 + 4 * 10

  // inline promotes static terms into inline
  dot(Vector(inline(2), inline(4)), Vector(inline(1), inline(10)))
    $\hookrightarrow$ 42
\end{listing}
\vspace{1.8mm}

Predictable partial evaluation has been a goal of many projects, however, to the best of our knowledge each solution has limitations:
\begin{itemize}
\item Programming languages Idris and D provide annotations \code{static} that can be placed on function arguments. In these systems static parameters are required to be deeply static and, thus, the \code{dot} function could not be expressed.
\item Type-directed partial evaluation~\cite{danvy1999type} and LMS~\cite{rompf2012lightweight} use types to communicate the programmers intent about partial evaluation and, thus, can express \code{dot} function. These approaches, however, require the programmer to implement two versions of \code{Vector} and other operations. This fosters, costly and hardly maintainable, code duplication.
\item MetaOCaml
\end{itemize}

Contributions:

Evaluation:

Sections:

\section{The \calculus Calculus}
\label{sct:calculus}

% TODO:
% Should we at all have a non-dynamic version?
\begin{figure*}[t]
\begin{multicols}{2}
\syntaxfig{
  t ::=                             & \lindent{\mbox{Terms:}}              \\
  \gap x,\ y                        & \mbox{identifier}                    \\
  \gap (x: \i{T}) \ra t             & \mbox{function}                      \\
  \gap t(t)                         & \mbox{application}                   \\
  \gap \{ \seq{x = t} \}            & \mbox{record}                        \\
  \gap t.x                          & \mbox{selection}                     \\
  \gap [X <: \i{T}] \ra t           & \mbox{type abstraction }             \\
  \gap t [\i{T}]                    & \mbox{type application}              \\
  \gap \inline{t}                   & \mbox{inline view}                   \\
  v ::=                             & \lindent{\mbox{Values:}}             \\
  \gap x \ra t                      & \mbox{function value}                \\
  \gap \{ \seq{x = t} \}            & \mbox{record value}                  \\
}
\syntaxfig{
  S,\ T,\ U ::=                     & \lindent{\mbox{Types:}}              \\
  \gap \i{S} \ra \j{T}              & \mbox{function type}                 \\
  \gap \{ \seq{x: \i{S}} \}         & \mbox{record type}                   \\
  \gap [X <: \i{S}] \ra \j{T}       & \mbox{universal type}                \\
  \gap Any                          & \mbox{top type}                      \\
  \i{T},\ \j{T},\ \k{T},\ \l{T} ::= & \lindent{\mbox{Binding-Time Types:}} \\
  \gap X                            & \mbox{type identifier}               \\
  \gap T,\ \dynamic{T}              & \mbox{dynamic type}                  \\
  \gap \static{T}                   & \mbox{static type}                   \\
  \gap \inline{T}                   & \mbox{inline type}                   \\
  \Gamma ::=                        & \lindent{\mbox{Contexts:}}           \\
  \gap \emptyset                    & \mbox{empty context}                 \\
  \gap \Gamma,\ x: iT               & \mbox{term binding}                  \\
  \gap \Gamma,\ X <: iT             & \mbox{type binding}                  \\
}
\end{multicols}
\caption{Syntax of \calculus}
\end{figure*}

We formalize the essence of our inlining system in a minimalistic calculus based
on $F_{<:}$ with lazy records. To accommodate predictable partial evaluation we
introduce binding-time annotations into the type system as first-class types that
represent three kinds of bindings:

\begin{enumerate}
  \item \textbf{Dynamic binding}. These are the types which express computation at runtime.
        All types written in the end user code are considered to be dynamic by default if
        no other binding-time annotation is given.

  \item \textbf{Static binding}. Values of static terms can be computed at compile-time
        (\eg constant expressions) but at are still evaluated at runtime by default.
        All language literals are static by default.

  \item \textbf{Inline binding}. And finally the types that correspond to terms that
        are hinted to be computed at compile-time whenever possible.
\end{enumerate}

\subsection{Composition}

An interesting consequence of encoding of binding times as first-class types is
ability to represent values which are partially static and partially dynamic.

For example lets have a look at simple record that describes a complex number with
two possible representations encoded through $isPolar$ flag:

\begin{equation}\nonumber
    complex: \static{\{ isPolar: \static{Boolean},\ a: Double,\ b: Double \}} \in \Gamma
\end{equation}

This type is constructed out of a number of components with varying binding times.
Representation encoding is known in advance and is static according to the signature.
Coordinates $a$ and $b$ do not have any binding-time annotation meaning that they are
dynamic.

Given this binding to $complex$ in our environment $\Gamma$ we can use $inline$ to obtain
a compile-time view to evaluate acess to $isPolar$ field at compile-time:
\begin{equation}\nonumber
  \inline{complex.isPolar}: \inline{Boolean}
\end{equation}

Any statically known expression can be promoted via $inline$. Selection of dynamic fields
on the other hand will return dynamic values despite the fact that record is statically known.
In practice this can be used to specialize a particular execution path in the application to
a particular representation by selectively inlining statically known parts.

Once you have inline view of the term it's also possible to demote it back to runtime evaluation
through $dynamic$ view.

Not all type and binding time combinations are correct though. We restrict types
to disallow nesting of more specific binding times into less specific ones.

% TODO Discuss:
%  - W-TAbs is liberal. It leaves possibility for un-inhibited types:
%    inline[X <: static Any] => dynamic(X => X) // is not inhabitable
%    inline[X <: static Any] => dynamic Int     // would be disallowed and it is a valid type
% This is taken care of type application later.
%  - Discuss with Sandro the problems with monotonicity.
\begin{figure}[H]
  \infax[\textsc{W-Any}]
  {\wff{\i{Any}}}

  \infrule[\textsc{W-Abs}]
  {i <: j \ \ \ \ i <: k \ \ \ \ \wff{\j{T_1}} \ \ \ \ \wff{\k{T_2}}}
  {\wff{\i{(\j{T_1} \ra \k{T_2})}}}

  \infrule[\textsc{W-TAbs}]
  {i <: j \ \ \ \ i <: k \ \ \ \ \wff{\j{S}} \ \ \ \ \wff{\k{T}}}
  {\wff{\i{([X <: \j{S}] \ra \k{T})}}}

  \infrule[\textsc{W-Rec}]
  {\forall j. \ \ \ \ i <: j \ \ \ \ \seq{\wff{\j{T}}}}
  {\wff{\i{\{\seq{x: \j{T}}\}}}}
\caption {Well formed types $\wff{\i{T}}$}
\end{figure}

This restiction allows us to reject programs that have inconsistent annotations.
For example the following function has incorrectly annotated parameter binding time:
\begin{equation}\nonumber
    (x: \inline{Int}) \ra x + 1
\end{equation}

This is inconsistent because the body of the function might not be evaluated at compile-time
(as the function is not inline.) As described in (\textsc{W-Abs}) functions may only have parameters
that are at most as specific as the function binding-time. In our example this doesn't hold as $inline$
is more specific than implicit $static$ annotation on function literal.

\subsection{Subtyping}

Another notable feature of our binding-time analysis system is deep integration with subtyping.
We believe that such integration is crucial for an object-oriented language that wants to
incorporate partial evaluation.

At core of the subtyping relation we have a subtyping relation on binding-time information
with $dynamic$ as top binding-time.

\begin{figure}[H]
\begin{multicols}{2}
  \infax[\textsc{I-Dynamic}]{i <: dynamic}

  \infax[\textsc{I-Static1}]{static <: static}

  \infax[\textsc{I-Static2}]{inline <: static}

  \infax[\textsc{I-Inline}]{inline <: inline}
\end{multicols}
\caption{Binding-time subtyping.}
\end{figure}

We proceed by threading binding time information throughout regular $F_{<:}$ subtyping rules
augmented with standard record types.

\begin{figure}[H]
  \infax[\textsc{S-Top}]
  {\Gamma \tsv \i{S} <: Any}

  \infax[\textsc{S-Refl}]
  {\Gamma \tsv \i{S} <: \i{S}}

  \infrule[\textsc{S-Trans}]
  {\Gamma \tsv \i{S} <: \j{U} \ \ \ \ \Gamma \tsv \j{U} <: \k{T}}
  {\Gamma \tsv \i{S} <: \j{U}}

  \infrule[\textsc{S-Inline}]
  {i <: j \ \ \ \ \Gamma \tsv S <: T}
  {\Gamma \tsv \i{S} <: \j{T}}

  \infrule[\textsc{S-TVar}]
  {X <: \i{T} \in \Gamma}
  {\Gamma \tsv X <: \i{T}}

  \infax[\textsc{S-Width}]
  {\{ x_p: i_p T_p\ ^{p \in 1..n+m} \} <: \{ x_p: i_p T_p\ ^{p \in 1..n} \}}

  \infrule[\textsc{S-Arrow}]
  {\Gamma \tsv \k{T_1} <: \i{S_1} \ \ \ \ \Gamma \tsv \j{S_2} <: \l{T_2}}
  {\Gamma \tsv \i{S_1} \ra \j{S_2} <: \k{T_1} \ra \l{T_2}}

  \infrule[\textsc{S-Depth}]
  {\forall p \in 1..n.\ i_p S_p <: j_p T_p}
  {\{ x_p: i_p S_p\ ^{p \in 1..n}\} <: \{ x_p: j_p T_p\ ^{p \in 1..n} \}}

  \infrule[\textsc{S-All}]
  {\Gamma,\ X <: \i{U_1} \tsv \j{S_2} <: \k{T_2}}
  {\Gamma \tsv [X <: \i{U_1}] \ra \j{S_2} <: [X <: \i{U_1}] \ra \k{T_2}}

  \infrule[\textsc{S-Perm}]
  {\{ x_p : i_p S_p\ ^{p \in 1..n}\} $ is permutation of $ \{ y_p: j_p T_p\ ^{p \in 1..n} \}}
  {\{ x_p : i_p S_p\ ^{p \in 1..n}\} <: \{ y_p: j_p T_p\ ^{p \in 1..n} \} }
\caption{Subtyping.}
\end{figure}

Integration between binding-time subtyping and subtyping on regular types is expressed through
(\textsc{S-Inline}) rule that merges the two into one coherent relation on binding-time types.

\subsection{Generics}

Crucial consequence of our design choices made in the system manifests in ability to use
regular generics as means to abstract over binding-time without any additional language constructs.

For example given a generic identity function:
\begin{equation}\nonumber
  identity: \static{([X <: Any] \ra \static{(X \ra X)})} \in \Gamma
\end{equation}

We can instantiate it to both in static and dynamic contexts through corresponding
type application:
\begin{align}\nonumber
  identity[\static{Int}] &: \static{(\static{Int} \ra \static{Int})} \\
  identity[Int]          &: \static{(Int \ra Int)}
\end{align}

In practice this allows us to write code that is polymorphic in the binding time without
any code duplication which is quite common in other partial evaluation systems.

This is possible due to the fact that we've integrated binding time information into types
and augmented subtyping relation with subtyping

\subsection{Typing}

To enforce well-formedness of types in a context of partial evaluation we customize
standard typing rules with additional constraints with respect to binding time.

% Questions/TODOs:
%  - Discussion: inline/static in the term position instead as a superscript on operations.
%     - examine two level lambda calculus.
%  - Introduce a lemma that shows that after reduction types remain well-formed!
%    - For each well formed type, all nested types have inlinity greater than the type.
%    - After a step of evaluation all types are well formed. Do we need this lemma? Is it subsumed by the preservation?
%    - Progress: Well typed term t is either a value or is further reducible.
%    - Term if t: T reduces to t' then t': T
\begin{figure}[H]
  \infrule[\textsc{T-Ident}]
  {x: \i{T} \in \Gamma}
  {\Gamma \tsv x: \i{T}}

  \infrule[\textsc{T-Rec}]
  {\forall t. \ \ \ \ \Gamma \tsv t: \j{T} \ \ \ \ \wff{\i{\{x: \j{T}\}}}}
  {\Gamma \tsv \i{\{ \seq{x = t} \}} : \i{\{\seq{x: \j{T}}\}}}

  \infrule[\textsc{T-App}]
  {\Gamma \tsv t_1: \i{(\j{T_1} \ra \k{T_2})} \ \ \ \ \Gamma \tsv t_2: \j{T_1}}
  {\Gamma \tsv t_1 (t_2) : \k{T_2}}

  \infrule[\textsc{T-Sel}]
  {\Gamma \tsv t: \i{\{ x = \j{T_1}, \seq{y = \k{T_2}} \}}}
  {\Gamma \tsv t.x : \j{T_1}}

  \infrule[\textsc{T-Inline}]
  {t $ is not literal$ \ \ \ \ \Gamma \tsv t: \static{T}}
  {\Gamma \tsv \inline{t}: \inline{T}}

  \infrule[\textsc{T-Dynamic}]
  {t $ is not literal$ \ \ \ \ \Gamma \tsv t: \i{T}}
  {\Gamma \tsv \dynamic{t}: \dynamic{T}}

  \infrule[\textsc{T-Sub}]
  {\Gamma \tsv t: \i{S} \ \ \ \ \Gamma \tsv \i{S} <: \j{T}}
  {\Gamma \tsv t: \j{T}}

  \infrule[\textsc{T-Func}]
  {\Gamma,\ x: \j{T_1} \tsv t: \k{T_2} \ \ \ \ \wff{\i{(\j{T_1} \ra \k{T_2})}}}
  {\Gamma \tsv \i{((x: \j{T_1}) \ra t)} : \i{(\j{T_1} \ra \k{T_2})}}

  \infrule[\textsc{T-TAbs}]
  {\Gamma,\ X <: \j{T_1} \tsv t_2: \k{T_2} \ \ \ \ \wff{\i{([X <: \j{T_1}] \ra \k{T_2})}}}
  {\Gamma \tsv \i([X <: \j{T_1}] \ra t_2): \i{([X <: \j{T_1}] \ra \k{T_2})}}

  \infrule[\textsc{T-TApp}]
  {\Gamma \tsv t_1: \i{([X <: \j{T_{11}}] \ra \k{T_{12}})}  \ \ \ \ \Gamma \tsv \l{T_2} <: \j{T_{11}}   \ \ \ \  \Gamma \tsv \i{} <: \l{}}
  {\Gamma \tsv t_1[\l{T_2}] : [X \mapsto \l{T_2}] \k{T_{12}}}
\caption{Typing.}
\end{figure}

The most significant changes lie in:
\begin{itemize}
  \item Additional checks in literal typing that ensure that constructed
        values correspond to well-formed types (\textsc{T-Func, T-Rec, T-TAbs}).
        To do this we typecheck literals together with possible binding-time term
        that might enclose it.
  \item New typing rules for binding-time views (\textsc{T-Inline, T-Dynamic}).
        These rules only cover non-literal terms as composition of binding-time view
        and literal itself is handled in corresponding typing rule for given literal.
\end{itemize}

\subsection{Partial Evaluation}

In order to simplify partial evaluation rules we erase all of the type information before partial evaluation.
This means that all functions become function values, type abstraction and application are complete eliminated.

% TODO rules do not capture the case when t_1 in t_1(t_2) is non-trivial inline. (PE-Sel, PE-App)
% TODO inline inline t ?
% TODO how to evaluate staitc terms. Counter examples:
%  - 1) Can there be dynamic rubish inside?
%  - 2) Problem here is that you can not really just call eval! Must be an eval with a context!
\begin{figure}[H]
  \infrule[\textsc{PE-Func}]
  {t \pe t'}
  {x \ra t \pe x \ra t'}

  \infrule[\textsc{PE-Rec}]
  {\seq{t} \pe \seq{t'}}
  {\{ \seq{x = t} \} \pe \{ \seq{x = t'} \}}

  \infrule[\textsc{PE-App}]
  {t_1 \pe t_1' \ \ \ \ t_1' \neq \inline{x \ra t} \ \ \ \ t_2 \pe t_2'}
  {t_1(t_2) \pe t_1'(t_2')}

  \infrule[\textsc{PE-IApp}]
  {t_1 \pe \inline{x \ra t} \ \ \ \ t_2 \pe t_2' \ \ \ \ [x \mapsto t_2'] t \pe t'}
  {t_1(t_2) \pe t'}

  \infrule[\textsc{PE-Sel}]
  {t \pe t' \ \ \ \ t' \neq \inline{x \ra t}}
  {t.x \pe t'.x}

  \infrule[\textsc{PE-ISel}]
  {t \pe \inline{\{x = t_x,\ \seq{y = t_y}\}} \ \ \ \ t_x \pe t_x'}
  {t.x \pe t_x'}

  \infrule[\textsc{PE-Inline}]
  {t $ is not literal$ \ \ \ \ t \pe t' \ \ \ \ t' \e v }
  {\inline{t} \pe \inline{v}}
\caption{Partial evaluation $t \pe t'$}
\label{fig:partial-evaluation}
\end{figure}

\subsection{Evaluation}
Once partial evaluation is complete we strip all binding-time terms and use regular untyped
lambda calculus evaluation rules extended with lazy records.

\begin{figure}[H]
  \infax[\textsc{E-Value}]
  {v \e v}

  \infrule[\textsc{E-App}]
  {t_1 \e x \ra t \ \ \ \ t_2 \e v \ \ \ \ [x \mapsto v] t \e v'}
  {t_1(t_2) \e v'}

  \infrule[\textsc{E-Sel}]
  {t \e \{ x = t_x,\ \seq{y = t_y}\} \ \ \ \ t_x \e v}
  {t.x \e v}
\caption{Evaluation $t \e v$}
\end{figure}

\subsection{Conjectures}

\begin{enumerate}
  \item Progress.
  \item Preservation.
  \item Static terms are closed over statically bound variables.
  \item Inline terms will be replaced with canonical value of corresponding type after partial evaluation.
\end{enumerate}


\bibliographystyle{plain}
\bibliography{vjovanov-lib}

\end{document}



