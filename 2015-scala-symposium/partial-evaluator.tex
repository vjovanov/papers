\section{Compile-Time Views in Scala}
\label{sct:interface}

In this section we informally present \tool, a staging extension for Scala based on compile-time views.
 \tool is a compiler plugin that executes in a phase after the
 Scala type checker. The plugin takes as input typechecked Scala programs and uses
 type annotations~\cite{odersky_1996_putting} to track and verify information about the biding-time
 of terms. It supports only two stages of compilation: host language compile-time
 (types annotated with \code{@ct}) and host language run-time (unannotated code).

To the user, \tool exposes a minimal interface (\figref{fig:interface}) with
a single annotation \code{ct} and a single function \code{ct}.

\begin{figure}
\begin{listing}
package object scalact {
  final class ct extends StaticAnnotation

  @compileTimeOnly def ct[T](body: => T): T = ???
}
\end{listing}
\label{fig:interface}
\caption{Interface of \tool.}
\end{figure}

\smartparagraph{Annotation \code{ct}} is used on types~(\eg,
\code{Int@ct}) to promote them to their compile-time views. The
annotation is integrated in the Scala's type system and, therefore, can be
arbitrarily nested in different variants of types.

Since all operations on compile-time views are executed at compile time, non-generic
 method parameters and result types of compile-time views also become compile-time views. Table \ref{tbl:ct-type}
 shows how the \code{@ct} annotation can be placed on types and how it affects method
 signatures on annotated types.

\begin{table*}[t]
\caption{Compile-time views of types and their corresponding method signatures.}
\label{tbl:ct-type}
\centering
\begin{tabularx}{\linewidth}{ X X X X }
\toprule

  Annotated Type              & \ &  Type's Method Signatures                          &  \\
  \code{Int@ct}               & \ &  \code{+(rhs: Int@ct): Int@ct}                     &  \\
  \code{Vector[Int]@ct}       & \ &  \code{map[U](f: (Int => U)@ct): Vector[U]@ct}     &  \\
                              & \ &  \code{length: Int@ct}                             &  \\
  \code{Vector[Int@ct]@ct}    & \ &  \code{map[U](f: (Int@ct => U)@ct): Vector[U]@ct}  &  \\
                              & \ &  \code{length: Int@ct}                             &  \\
  \code{Map[Int@ct, Int]@ct}  & \ &  \code{get(key: Int@ct): Option[Int]@ct}           &  \\

\bottomrule
\end{tabularx}
\end{table*}

 In \tabref{tbl:ct-type}, on \code{Int@ct} both parameters and result types of all
 methods are also compile-time views. On the other hand, \code{Vector[Int]@ct} has parameters
 of all methods transformed except the generic ones. In effect, this, makes higher order combinators of \code{Vector}
 operate on dynamic values, thus, function \code{f} passed to \code{map} accepts
 the dynamic value as input. Type \code{Vector[Int@ct]@ct} has all methods executed
 at compile-time. The return type of the function \code{map} on \code{Vector[Int@ct]@ct}
 can still be either dynamic or a compile-time view due to the type parameter \code{U}.

Annotation \code{ct} can be used to achieve simple inlining of statically
 known methods and functions. This is achieved by putting the annotation of the method/function
 definition:\begin{lstparagraph}
 def dot[V: Numeric]
  (v1: Vector[V], v2: Vector[V]): V
\end{lstparagraph}
Annotated methods will have an annotated method type\begin{lstparagraph}
((v1: Vector[V], v2: Vector[V]) => V)@ct
\end{lstparagraph} which can not be written by the users. This is not the first time
that inlining is achieved through partial evaluation~\cite{monnier2003inlining}.



\smartparagraph{Function \code{ct}} is used at the term level
 for promoting literals, modules, and methods/functions into their compile-time views.
 Without \code{ct} we would not be able to instantiate compile-time views of types.
 \tabref{tbl:ct-term} shows how different types of terms are promoted to their
 compile-time views. An exception for promoting terms to compile-time views is the
 \code{new} construct. Here we use the type annotation on the constructed type.


\begin{table*}[t]
\caption{Promotion of terms to their compile-time views.}
\label{tbl:ct-term}
\centering
\begin{tabularx}{\linewidth}{ X X }
\toprule

  Promoted Term        \quad \quad \quad & Term's Type                      \\
  \code{ct(Vector)(1, 2, 3)            } & \code{: Vector[Int]@ct        }  \\
  \code{ct(Vector)(ct(1), ct(2), ct(3))} & \code{: Vector[Int@ct]@ct     }  \\
  \code{ct((x: Int@ct) => x)           } & \code{: (Int@ct => Int@ct)@ct }  \\
  \code{ct((x: Int) => x)              } & \code{: (Int => Int)@ct       }  \\
  \code{new (::@ct)(1, Nil)            } & \code{: (::[Int])@ct          }  \\
  \code{new (::@ct)(ct(1), ct(Nil))    } & \code{: (::[Int@ct])@ct       }  \\

\bottomrule
\end{tabularx}
\end{table*}

\subsection{Tracking Binding-Time of Terms}
\label{sct:static}

 Internally \tool has additional type annotations for tracking the binding time of terms.
  Type of each term is annotated with either \code{dynamic}, \code{static}, or \code{ct}. \code{dynamic} denotes
  that the term can only be known at runtime, \code{static} that the term is known
  at compile-time but it will not be computed at compile time, and \code{ct} that
  the term will be computed at compile-time.

 Tracking static terms was studied in the context of binding-time analysis
  in partial evaluation for typed~\cite{nielson_1988_automatic} and
  untyped~\cite{gomard1991partial} languages. We use similar techniques, however,
  unlike in partial evaluation we do not evaluate static terms at compile time. They are tracked for verifying
  correctness and providing convenient implicit conversions. Static terms are evaluated only
  when they are explicitly marked by the programmer with \code{ct}.

  % What are the static terms
In \tool language literals, functions, direct class constructor calls with static arguments, and static method
 calls with static arguments are marked as static. Examples of static terms are\begin{lstparagraph}
1.0, "1", (x: Int => x), new Cons(1, Nil), List(1,2,3)
\end{lstparagraph}

\subsection{Least Upper Bounds}
\label{sct:lub}

 We use subtyping of Scala to simplify tracking of binding times by introducing a
 subtyping relation between \code{dynamic}, \code{static}, and \code{ct}. We argue that
 a \code{static} type is a more specific \code{dynamic} as it is statically known
 and that \code{ct} is more specific than \code{static} as its operations are executed
 at compile time. Therefore we establish that\begin{lstparagraph}
                 ct <: static <: dynamic
\end{lstparagraph}

 The use of subtyping simplifies tracking binding times of terms as in all cases
 where least upper bounds are calculated we can use the same mechanism for binding-times.
 An interesting example are the binding times of type parameters:\begin{lstparagraph}
ct(List)(1, ct(2)): List[Int@static]@ct
ct(List)(ct(1), ct(2)): List[Int@ct]@ct
ct(List)((x: Int@dynamic), ct(2)): List[Int@dynamic]@ct
\end{lstparagraph}

Notable exception are control flow constructs for which the original Scala rules for least
 upper bounds do not hold. The binding-time of control flow constructs does not
 depend only on the return type of the branches but also on the conditionals. For example, if
 both branches of an \code{if} construct are \code{static} the result can still be \code{dynamic}
 if the condition is \code{dynamic}. Here subtyping also helps as the binding type of the
 return value is simply calculated as \code{lub(c, thn, elz)} where \code{lub(tps: Type*)} is a function
 for computing the least upper bounds of types, and \code{c}, \code{thn}, \code{elz} are respectively
 binding times of the condition, the then branch, and the else branch. The same principles can be applied for
 pattern matching.

\subsection{Well-Formedness of Compile-Time Views}
\label{sct:wf-ctv}

% Nice description of csp and pointer to the right paper.
Earlier stages of computation can not depend on values from later stages. This property,
 defined as \emph{cross-stage persistence}~\cite{taha_multi-stage_1997,westbrook2010mint},
 imposes that all operations on compile-time views must known at compile time.

To satisfy cross-stage persistence \tool verifies that binding time of composite
 types~(\eg, polymorphic types, function types, record types, etc.) is always
 a subtype of the binding time of their components. In the following example,
 we show malformed types and examples of terms that are inconsistent:\begin{lstparagraph}
xs: List[Int@ct]     => ct(Predef).println(xs.head)
fn: (Int@ct=>Int@ct) => ct(Predef).println(fn(ct(1)))
\end{lstparagraph}

In the first example the program would, according to the semantics of \code{@ct}, print a head of the list at compile time.
 However, the head of the list is known only in the runtime stage. In the second example the program should
 print the result of \code{fn} at compile time but the body of the function will
 be known only at runtime. By causality such examples are not possible.

% Examples on classes
On functions/methods the \code{ct} annotation requires that function/method bodies are known at compile-time.
 Otherwise, inlining of such functions/methods would not be possible at compile-time. In Scala,
 method bodies are statically known in objects and classes with final methods, thus, the \code{ct}
 annotation is only applicable on such methods.

\subsection{Implicit Conversions}
\label{sct:implicits}

% Requires desugaring
If method parameters require compile-time views of a type the corresponding arguments
 in method application would always have to be promoted to \code{ct}.
 In some libraries this could require an inconveniently large number
 of annotations.

To minimize the number of required annotations we introduce implicit conversions from certain \code{static} terms to \code{ct} terms.
 The conversions support translation of language literals, direct class constructor calls with static arguments, and static method
 calls with static arguments into their compile-time views. Since our compile-time evaluator does
 not use Asai's~\cite{asai2002binding,sumii2001hybrid} method to keep track of
 the value of each static term, we disallow implicit conversions of terms with static variables.

For example, for a factorial function \begin{lstparagraph}
def fact(n: Int @ct): Int@ct =
  if (n == 0) 1 else fact(n - 1)
 \end{lstparagraph} we will not require promotions of literals \code{0}, and \code{1}. Furthermore,
 the function can be invoked without promoting the argument into it's compile-time view:\begin{lstparagraph}
fact(5)
  $\hookrightarrow$ 120
 \end{lstparagraph}

Without implicit conversions the factorial functions would be more verbose \begin{lstparagraph}
def fact(n: Int @ct): Int@ct =
  if (n == ct(0)) ct(1) else fact(n - ct(1))
 \end{lstparagraph} as well as each function application (\code{fact(ct(5))}).
