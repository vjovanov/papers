\section{\tool Interface}
\label{sct:interface}

In this section we present \tool, a staging extension for Scala based polymorphic binding-time analysis.
 \tool is a compiler plugin that executes in a phase after the
 Scala type checker. The plugin takes as input typechecked Scala programs and uses
 type annotations~\cite{odersky_1996_putting} to track and verify information about the biding-time
 of terms. It supports only two stages of compilation: host language compile-time
 (types annotated with \code{@ct}) and host language run-time (unannotated code).

To the user, \tool exposes a minimal interface (\figref{fig:interface}) with
a single annotation \code{ct} and a single function \code{ct}.

\begin{figure}
\begin{listing}
package object scalact {
  final class ct extends StaticAnnotation

  def ct[T](body: => T): T = ???
}
\end{listing}
\label{fig:interface}
\caption{Interface of \tool.}
\end{figure}

\smartparagraph{Annotation \code{ct}} is used on types~(\eg,
\code{Int@ct}) to promote them to their compile-time views. The
annotation is integrated in the Scala's type system and, therefore, can be
arbitrarily nested in different variants of types.

Since operations on compile-time views should be executed at compile time, the \code{ct} type
can be viewed as the original type whose instance is known at compile time with additional
 methods whose non-generic method parameters and result types also become compile-time views (if possible).
 Generic parameters remain unchanged as their binding time is defined during corresponding
 type application. Table \ref{tbl:ct-type} shows how the \code{@ct} annotation can be
 placed on types and how it affects method signatures of additional methods. Note that methods are not
 in fact added to the type but the binding-time polymorphic behavior of original methods can be observed
 as additional methods.

\begin{table*}[t]
\caption{Compile-time views of types and additional methods that will be available to the user.}
\label{tbl:ct-type}
\centering
\begin{tabularx}{\linewidth}{ X X X X }
\toprule

  Annotated Type              & \ &  Type's Method Signatures                          &  \\
  \code{Int@ct}               & \ &  \code{+(rhs: Int@ct): Int@ct}                     &  \\
  \code{Vector[Int]@ct}       & \ &  \code{map[U](f: (Int => U)@ct): Vector[U]@ct}     &  \\
                              & \ &  \code{length: Int@ct}                             &  \\
                              & \ &  \code{hashCode: Int}                              &  \\
  \code{Vector[Int@ct]@ct}    & \ &  \code{map[U](f: (Int@ct => U)@ct): Vector[U]@ct}  &  \\
                              & \ &  \code{length: Int@ct}                             &  \\
                              & \ &  \code{hashCode: Int@ct}                           &  \\
  \code{Map[Int@ct, Int]@ct}  & \ &  \code{get(key: Int@ct): Option[Int]@ct}           &  \\

\bottomrule
\end{tabularx}
\end{table*}

 In \tabref{tbl:ct-type}, on \code{Int@ct} both parameters and result types of all
 methods are also compile-time views. On the other hand, \code{Vector[Int]@ct} has parameters
 of all methods transformed except the generic ones. In effect, this, makes higher order combinators of \code{Vector}
 operate on dynamic values, thus, function \code{f} passed to \code{map} accepts
 the dynamic value as input. Note that \code{hashCode} of \code{Vector[Int]@ct} returns a dynamic value--this is
 due to its implementation that internally operates on generic values that are dynamic.
 Type \code{Vector[Int@ct]@ct} has all additional methods promoted to compile-time. The return type of the function \code{map} on \code{Vector[Int@ct]@ct}
 can still be either dynamic or a compile-time view due to the type parameter \code{U}.

Annotation \code{ct} can also be used to achieve inlining of statically known methods and functions.
 This is achieved by putting the annotation of the method/function\footnote{This is not the first time
that inlining is achieved through partial evaluation~\cite{monnier2003inlining}.}
 definition:\begin{lstparagraph}
 @ct def dot[T: Numeric]
  (v1: Vector[T], v2: Vector[T]): T
\end{lstparagraph}
Annotated methods will have an annotated method type\begin{lstparagraph}
((v1: Vector[T], v2: Vector[T]) => T)@ct
\end{lstparagraph} which can not be written by the users.

% Binding time is inferred for type parameters.
\smartparagraph{Function \code{ct}} is used at the term level
 for promoting literals, modules, and methods/functions into their compile-time views.
 \tabref{tbl:ct-term} shows how different types of terms are promoted to their
 compile-time views. An exception for promoting terms to compile-time views is the
 \code{new} construct. Here we use the type annotation on the constructed type.

\begin{table*}[t]
\caption{Promotion of terms to their compile-time views.}
\label{tbl:ct-term}
\centering
\begin{tabularx}{\linewidth}{ X X }
\toprule

  Promoted Term        \quad \quad \quad & Term's Type                      \\
  \code{ct(Vector)(1, 2, 3)            } & \code{: Vector[Int]@ct        }  \\
  \code{ct(Vector)(ct(1), ct(2), ct(3))} & \code{: Vector[Int@ct]@ct     }  \\
  \code{ct((x: Int@ct) => x)           } & \code{: (Int@ct => Int@ct)@ct }  \\
  \code{ct((x: Int) => x)              } & \code{: (Int => Int)@ct       }  \\
  \code{new (::@ct)(1, Nil)            } & \code{: (::[Int])@ct          }  \\
  \code{new (::@ct)(ct(1), ct(Nil))    } & \code{: (::[Int@ct])@ct       }  \\

\bottomrule
\end{tabularx}
\end{table*}

\subsection{Well-Formedness of Compile-Time Views}
\label{sct:wf-ctv}

% Nice description of csp and pointer to the right paper.
Earlier stages of computation can not depend on values from later stages. This property---defined as \emph{cross-stage persistence}~\cite{taha_multi-stage_1997,westbrook2010mint}---imposes that all operations on compile-time views must be known at compile time.

To satisfy cross-stage persistence \tool verifies that binding time of composite
 types~(\eg, polymorphic types, function types, record types, etc.) is always
 a subtype of the binding time of their components. In the following example,
 we show malformed types and examples of terms that are inconsistent:\begin{lstparagraph}
xs: List[Int@ct]     => ct(Predef).println(xs.head)
fn: (Int@ct=>Int@ct) => ct(Predef).println(fn(ct(1)))
\end{lstparagraph}

In the first example the program would, according to the semantics of \code{@ct}, print a head of the list at compile time.
 However, the head of the list is known only in the runtime stage. In the second example the program should
 print the result of \code{fn} at compile time but the body of the function will
 be known only at runtime. By causality such examples are not possible.

% Examples on classes
On functions/methods the \code{ct} annotation requires that function/method bodies are known at compile-time.
 Otherwise, inlining of such functions/methods would not be possible at compile-time. In Scala,
 method bodies are statically known in objects and classes with final methods, thus, the \code{ct}
 annotation is only applicable on such methods.

\subsection{Minimizing the Number of Annotations}
\label{sct:implicits}

One of the design goals of \tool is to minimize the number of superfluous staging annotations. We achieve
 this by implicitly adding annotations and term promotions that would with high probability be added by the user.

{\bf Adding \code{ct} to Functions.} Due to cross-stage persistence if a single parameter of
a function is annotated with \code{ct} the whole function must also be \code{ct}. Since forgetting
this annotation would only result in an error we implicitly add
the \code{ct} annotation when at least one parameter type or the result type is marked as \code{ct}.

{\bf Implicit Conversions.} If method parameters require the compile-time view of a type the corresponding arguments
 in method application would always have to be promoted to \code{ct}.
 In some libraries this could require an inconveniently large number of annotations.

To minimize the number of required annotations we introduce \emph{implicit conversions}
 from certain \code{static} terms to \code{ct} terms. Note that we use a custom mechanism for
 implicit conversions based on type annotations. This is necessary as Scala implicit conversions
 are oblivious to type annotations (\sct{sct:limitations}).

The conversions support translation of language literals, direct class constructor calls with static arguments, and static method
 calls with static arguments into their compile-time views. Since our compile-time evaluator does
 not use Asai's~\cite{asai2002binding,sumii2001hybrid} method to keep track of
 the value of each static term, we disallow implicit conversions of terms with static variables.

For example, for a factorial function \begin{lstparagraph}
def fact(n: Int@ct): Int@ct =
  if (n == 0) 1 else fact(n - 1)
 \end{lstparagraph} we will not require promotions of literals \code{0}, and \code{1}. Furthermore,
 the function can be invoked without promoting the argument into it's compile-time view:\begin{lstparagraph}
fact(5)
  $\hookrightarrow$ 120
 \end{lstparagraph}

Without implicit conversions the factorial functions would be more verbose
\begin{lstparagraph} @ct def fact(n: Int@ct): Int@ct =   if (n == ct(0)) ct(1)
else fact(n - ct(1))  \end{lstparagraph} as well as each function application
(\code{fact(ct(5))}).

Implicit conversions are safe in all cases except when the user should be warned about potential
code explosion. For example, a user could accidentally call \code{fact} with a very large number and
cause code explosion without even knowing that staging is happening. If \code{ct} was required
it would remind the users about the potential problems. In the design of \tool we decided to
prefer less annotations over code-explosion prevention.


\section{Polymorphic Binding-Time Analysis} \label{sct:bta}

Scala is an object oriented language with numerous features and separate compilation. To avoid a mismatch between
the formalization and Scala we describe binding-time analysis on the minimal subset of Scala that gives insights in the
actual implementation.
\newcommand{\alt}{\mid}
\newcommand{\ct}[1]{ct \; #1}
\begin{align*}
\shortintertext{Terms:}
t       &::= t\;t \alt \func{x: \tau}{t} \alt x[\overline{\tau}] \alt x \alt \ct{v} \\
v       &::= \func{x: \tau}{t} \alt c\\
c       &::= \ktrue \alt \kfalse \alt 0 \alt 1 \alt \dots\\
\shortintertext{Types:}
b       &::= \beta \alt  rt \alt  ct \\
\gamma  &::= \forall \overline{\beta} \Rightarrow C. \gamma \alt \sigma'\\
\sigma' &::= \tau \alt \sigma \alt \gamma\langle{\overline{b}}\rangle\\
\sigma  &::= \univ{\overline{X^b}}{\tau}\\
\tau    &::= X^b \alt \tau \rightarrow^b \tau \alt \sigma[\overline{\tau}]\alt \btyp\\
\btyp   &::= \kbool^b \alt \kint^b
\end{align*}

Well formed types:
\begin{figure}[H]
  \infrule
  {\wff(\sigma) | b \ \ \ \ \wff(\overline{\tau}) | \overline{\tau} <: b}
  {\wff{\sigma[\overline{\tau}] | b}}}

  \infrule
  {\wff(\tau_1) | b_1 \ \ \ \ \wff(\tau_2) | b_2 \ \ \ \ b_1<:b_3 \ \ \ \ b_2<:b_3}
  {\wff{\i{(\tau_1 \rightarrow^{b_3} \tau_2) | b_3}}}

  \infrule
  {\wff(\sigma) | b \ \ \ \ \wff(\overline{\tau}) | \overline{\tau} <: b}
  {\wff{\sigma[\overline{\tau}] | b}}}

  \infax
  {\wff(X^b)|b}

  \infax
  {\wff(Bool^b)|b}

  \infax
  {\wff(Int^b)|b}

\caption {Well formed types $\wff{\i{T}}$}
\end{figure}

Coaercions:

Subtyping relation:

Deviate from regular underline/overline
Context:

Constructors:
 % $$\context class tname: \lambda \overline{X^{rt}}.\func{x: \tau}{t}$$
Methods:
 % $$\context def mname: \lambda \overline{X^{rt}}.\func{x: \tau}{t}$$

Translation to polymorphic calculus:

generic translations rules


To achieve polyvariant staging we augment Scala with \emph{binding-time
annotations}. There are two concrete binding time annotations $rt$ and $ct$
(often named $D$ and $S$ in related work). Annotations $rt$ and $ct$ represent
run-time binding time and compile time binding time respectively. To achieve
binding-time polymorphism we also introduce binding time variables $b$.

Binding times form a two-point lattice with the ordering operator $<:$.
TODO
% Maybe define the rules.
Binding time variables are introduced with binding time abstraction $\forall \overline{\beta}$ where $\overline{\beta}$
stands for a set of variables. Free binding time variables are tracked in a context $B$.

To achieve polyvariant staging and cross stage persistence we introduce a set $C$ of constraints
on binding time variables. The set $C$ contains assumptions on binding-time variables defined as $B, C \tsv \beta <: b$ where all free variables are in context $B$. Finally, a polyvariant term is defined as:

$$B, C \tsv \forall \overline{\beta} \Rightarrow \{\overline{\beta_i <: b}\}. \; t$$

The corresponding application of polymorphic terms is defines as

 $$t<\overline{b}>$$

{\bf Types.} To preserve clarity we will look only at a subset of Scala types that includes functions, rank-1 parametric polymorphism, and base types:

We distinguish our selves from the others by keeping rank-1 and by annotating type parameters separately
Mention that for every type variable we have a corresponding binding time variable.
{\bf Well formedness.} For each type we want to assure two things: \emph{i)} cross-stage persistence, \emph{ii)} that for each type variable we have a corresponding binding time variable. We assure the cross stage persistence
$$C contains wff(\sigma)$$
$$wff(\tau \rightarrow \tau) = \cup \cup$$
$$wff(\tau[\tau]) = \cup \cup$$
$$wff(\forall\alpha[\tau]) = \cup \cup$$

$$$$


% Define binding time abstractions

% Define polymorphic binding times

% Define the translation

\subsection{Nominal Types and Subtyping}
\label{sct:nominal-types}

\subsection{Compile-Time Evaluation}
\label{sct:evaluation}

\subsection{Mutable State}
\label{sct:mutable-state}
% TODO arrays and variables
% TODO attachments
% TODO mutable state not for now
% TODO emulate arrays as mutable lists
