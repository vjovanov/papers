\section{Evaluation}
\label{sct:evaluation}

In this section we evaluate the amount of code that is obviated with \tool compared to existing
type directed staging systems (\sct{sct:duplication}). Then we evaluate performance of
\tool compared to LMS and hand optimized code (\sct{sct:performance})

\subsection{Reduction in Code Duplication}
\label{sct:duplication}

Evaluating reduction of duplicated code (for reification and code generation) in type based
staging systems is difficult as the factor varies from program to program. To avoid benchmark dependent
results we instead calculate the lower bound on the duplication factor.

Given that we have a method on a type \code{T} whose body contains \code{n} lines of code (without
the method definition). To introduce the same method on an annotated type \code{Rep[T]} we need another
 method for reification which has at least 1 line of code. Then we need code generation
 logic, which, if we use the same language should not have less lines than the original method
 plus at least one line for matching the reified method. For method of $n$ lines
 we get a lower bound on the code duplication factor of:$$
 2n+3/n+1
$$
For single line methods ($n=0$) the factor is 3 and for large methods ($n\rightarrow\infty$) it converges to 2.

\subsection{Performance of Generated Code}

In this section we compare performance of \tool with LMS and original code. All benchmarks
are executed on an Intel Core i7 processor (4960HQ) working frequency of 2.6 GHZ with 16GB
of DDR3 with a working frequency of 1600 MHz. For all benchmarks we use Scala
2.11.5 and the HotSpot(TM) 64-Bit Server (24.51-b03) virtual machine. In all benchmarks
the virtual machine is warmed up, no garbage collection happens, and all reported numbers are
a mean of 5 measurements.

In \tabref{tbl:numbers} we show execution time normalized to original code for:
 \emph{i)} \code{pow(42.0, 10)},
 \emph{ii)} \code{min(a, b, c, d, e)},
 \emph{iii)} inner product of two statically known vectors of size 50,
 and \emph{iv)} the butterfly network of size 4 for fast Fourier transform \code{fft} (equivalent to code presented by Rompf and Odersky~\cite{rompf2012lightweight}). For all benchmarks the performance results are equivalent to LMS.

\label{sct:performance}
\begin{table}[h]
\caption{Speedup of LMS and \tool compared to the naive implementation of the algorithms.}
\label{tbl:numbers}
\centering
\begin{tabularx}{\linewidth}{ X X X }
\toprule

  Benchmark                   &  LMS      &  \tool                             \\
  \code{pow}                  &    221.75 & 221.70                             \\
  \code{min}                  &    1.82   & 1.79                               \\
  \code{dot}                  &  246.08   & 246.08                             \\
  \code{fft}                  &  12.14    & 12.88                              \\

\bottomrule
\end{tabularx}
\end{table}
