\section{Evaluation}
\label{sct:evaluation}

In this section we evaluate the amount of code that is obviated compared to existing
type directed staging systems (\sct{sct:duplication}). Then we evaluate performance of
\tool compared to LMS and hand optimized code (\sct{sct:performance})

\subsection{Reduction in Code Duplication}
\label{sct:duplication}

Evaluating reduction of duplicated code (for reification and code generation) in type based
staging systems is difficult as the factor varies from program to program. To avoid benchmark dependent
results we instead calculate the lower bound on the duplication factor.

Given that we have a method on a type \code{T} whose body contains \code{n} lines of code (without
the method definition). To introduce the same method on an annotated type \code{Rep[T]} we need another
 method for reification which has at least 1 line of code. Then we need code generation
 logic, which, if we use the same language should not have less lines than the original method
 plus at least one line for matching the reified method. For method of $n$ lines
 we get a lower bound on the code duplication factor of:$$
 2n+3/n+1
$$
For single line methods ($n=0$) the factor is 3 and for large methods ($n\rightarrow\infty$) it converges to 2.

\subsection{Performance of Generated Code}
\label{sct:performance}
\begin{table}[h]
\caption{Comparison of \tool with LMS and hand optimized code.}
\label{tbl:numbers}
\centering
\begin{tabularx}{\linewidth}{ X X X X}
\toprule

  Benchmark                  & \tool                                             &  LMS      &  Optimized by Hand          \\
  \code{pow}                 &                                                   &           &                             \\
  \code{min}                 &                                                   &           &                             \\
  \code{dot}                 &                                                   &           &                             \\
  \code{fft}                 &                                                   &           &                             \\

\bottomrule
\end{tabularx}
\end{table}
