\section{Discussion}
\label{sct:discussion}

% State the three possibilities.
To distinguish terms executed at compile-time from terms executed at runtime with type annotations we have the following possibilities:
\begin{enumerate}
\item Annotate types of all terms that should be executed at runtime. Here all types analyzed LMS and realized that this is not an option.

\item Annotate types of terms that should be executed at runtime but introduce scopes (e.g., method bodies) for which this rule applies.
In this way we would avoid annotating types of all runtime terms. This approach is taken by MacroML where
macro functions are executed at compile time and quoted terms are executed at runtime. First approach is, also,
a special case of this approach where there is a single scope for the whole language.

\item Annotate types of terms that are executed at compile time. This approach is used with \tool and annotated types are
called compile-time views.
\end{enumerate}

To compare approach of \tool with the first approach we analyzed 817 functions
from the OptiML~\cite{sujeeth_optiml:_2011} DSL based on LMS. With the \tool scheme
OptiML would require more than 2x less annotations to implement.

Compared to the second approach our solution is simpler to comprehend and communicate. In the second approach there are
two things that users need to understand when reasoning about staged programs: \emph{i)} where does
the compile time scope start, and \emph{ii)} which terms are annotated. With \tool the comprehension
is simple: terms whose types are annotated with \code{ct} are executed at compile time.

% The number of annotations if it is a mostly manipulate code snippets.
It is also interesting to the second and third approaches. Here the number of annotations
depends on the program. If the programs are mostly partially evaluated the second approach
is better. These category of programs could also be regarded as code generators as most of the code
is executed at compile time and produces large outputs. When programs are comprised of mostly
runtime values the approach of \tool requires less annotations.


