\section{Introduction}
\label{sct:introduction}

 \emph{Multi-stage programming} (or \emph{staging}) is a meta-programming technique
  where compilation is separated in multiple \emph{stages}. Execution of each
  stage outputs code that is executed in the next stage of compilation. The first
  stage of compilation happens at the \emph{host language} compile time, the second
  stage happens at the host language runtime, the third stage happens at runtime of
  runtime generated code, etc. Different stages of compilation can be executed in the same
  language~\cite{taha_multi-stage_1997,nielson2005two} or in different languages~\cite{brown_heterogeneous_2011,devito2013terra}.
  In this work we will focus on staging systems where all stages are in the same language and that, through static typing, assure
  that terms in the next stage are well typed.

  Notable staging systems in statically typed languages are
  MetaOCaml~\cite{taha_multi-stage_1997,calcagno2003implementing}
  and LMS~\cite{rompf2012lightweight}. These systems were successfully applied as a
  \emph{partial evaluatior}~\cite{jones1993partial}: for removing abstraction
  overheads in high-level programs~\cite{carette2005multi,rompf2012lightweight},
  for domain-specific languages~\cite{czarnecki_dsl_2004,jonnalagedda2014staged,taha2004gentle}, and for converting language
  interpreters into compilers~\cite{lancet,futamura1999partial}. Staging originates
  from research on two-level~\cite{nielson2005two,davies1996temporal} and multi-level~\cite{davies1996modal} calculi.

 We show an example of how staging is used for partial evaluation of a function
 for computing the inner product of two vectors\footnotemark[1]:\begin{lstparagraph}
def dot[T:Numeric](v1: Vector[T], v2: Vector[T]): T =
  (v1 zip v2).foldLeft(zero[T]) {
    case (prod, (cl, cr)) => prod + cl * cr
  }
 \end{lstparagraph}

In function \code{dot}, if vector sizes are constant, the inner product can
 be partially evaluated into a sum of products of vector components. To achieve partial evaluation,
 we must communicate to the staging system that operations on values of vector components
 should be executed in the next stage. The compilation stage
 in which a term is executed is determined by \emph{code quotation}~(in MetaOCaml)
 or by parametric types \code{Rep}~(in LMS). In LMS marking
 that the vector size is statically known is achieved by annotating only vector elements with
 a \code{Rep} type\footnotemark[2]:\begin{lstparagraph}
def dot[T:Numeric]
  (v1: Vector[Rep[T]], v2: Vector[Rep[T]]): Rep[T]
 \end{lstparagraph}

Here the \code{Rep} annotations on \code{Rep[T]} denote that elements of vectors will be known
 only in the next stage (in LMS, this is a stage after run-time compilation). After run-time compilation \code{zip},
 \code{foldLeft}, and pattern matching inside the closure will not exist
 as they were evaluated in the previous stage of compilation (host language runtime). Note that in
 LMS unannotated code is always executed during host-language runtime
 and type-annotated code is executed after run-time compilation.

{\bf Staging at host language compile time.} How can we use staging for programs whose values are statically
 known at the host language compile-time (the first stage)? Existing staging frameworks
 treat unannotated terms as runtime values of the host language and annotated terms as
 values in later stages of compilation. Even if we would take that the first stage is executed
 at the host language compile time, we would have to annotate all run-time values.
 Annotating all values is cumbersome since host language run-time values comprise
 the majority of user programs~(\sct{sct:discussion}).

\footnotetext[1]{All code examples are written in \emph{Scala}. It is necessary to
know the basics of Scala to comprehend this paper.}
\footnotetext[2]{In this work we use LMS as a representative of type-based staging systems.}

MacroML~\cite{ganz2001macros} expresses macros as two-stage computations that start executing from host language compile time.
 In MacroML, parameters of macros can be annotated as an early stage computation. These parameters
 can then used in escaped terms for compile-time computation. Terms scheduled for runtime execution,
 withing the escaped terms, again need to be quoted with brackets. This,
 in effect, imposes quotation for both escaping and brackets which requiring additional effort.

% Type annotations, and thus all operations continue to work!
{\bf Code Duplication.} Staging systems based on type annotations (e.g., LMS and type-directed
partial evaluation~\cite{danvy1999type}) inherently require code duplication as,
a priory, no operations are defined on \code{Rep} annotated types. For example,
in the LMS version of the \code{dot} function, all numeric types (\ie, \code{Rep[Int]}, \code{Rep[Double]}, etc.)
must be re-implemented in order to typecheck the programs and achieve code generation
for the next stage.

Sujeeth et al.~\cite{forge} and Jovanovic et al.~\cite{yin-yang}
 propose generating code for the next stage computations based on
 a language specification. These approaches solve the problem,
 but they require writing additional specification for the libraries,
 require a large machinery for code generation,
 and support only restricted parts of Scala.

{\bf Annotating the Previous Stage.} The main idea of this paper is that \emph{annotated types}
 should denote computations that happen during the \emph{previous stage} of compilation.
 The reason is that static terms appear less frequently than run-time terms in a large set
 of analyzed programs (\sct{sct:discussion}). Therefore, annotating static terms
 introduces less overhead for the programmer.

% TODO count the number of terms
We treat annotated types as \emph{compile-time views}
 of existing data types. Compile-time view of a type denotes that all operations on that type are
 executed at host language compile time. We promote types to their compile-time views
 with the \code{@ct} annotation (\eg, \code{Int@ct}). Similarly, statically known terms can be promoted
 their compile time duals with the \code{ct} function on the term level.
 By having two views of the same type we obviate the need for introducing reification and code generation logic for
 existing types.

% Restricted to compile-time, but just techincal.
With compile-time views, to require that vectors \code{v1} and \code{v2} are
 static and to partially evaluate the function, a programmer needs to make
 a simple modification of the \code{dot} signature:\begin{lstparagraph}
def dot[V: Numeric@ct]
  (v1: Vector[V]@ct, v2: Vector[V]@ct): V
\end{lstparagraph}

Since, vector elements are polymorphic the result
 of the function can be a dynamic value, or a compile-time view
 that can be further used for compile-time computations. The binding time of the return type
 of \code{dot} will match the binding time of vector elements:

\vspace{1.8mm}
\begin{listing}[mathescape]
  // [el1, el2, el3, el4] are dynamic decimals
  dot(Vector(el1, el2), Vector(el3, el4))
    $\hookrightarrow$ (el1 * el3 + el2 * el4): Double

  // ScalaCT tracks wheater terms are static
  dot(Vector(2.0, 4.0), Vector(1.0, 10.0))
    $\hookrightarrow$ (2.0 * 1.0 + 4.0 * 10.0): Double@static

  // ct promotes static terms to compile-time views
  dot(Vector(ct(2), ct(4)),
      Vector(ct(1), ct(10)))
    $\hookrightarrow$ 42: Double@ct
\end{listing}
\vspace{1.8mm}

In this paper we contribute to the state of the art:
\begin{itemize}

 \item By introducing compile-time views~(\sct{sct:interface}) as means to succinctly achieve
  type safe two-stage programming starting from host language compile time.

 \item By obviating the need for reification and code generation logic in type based staging systems.

 \item By demonstrating the usefulness of compile-time views in four case
  studies (\sct{sct:case-studies}): inlining, partially evaluating recursion,
  removing overheads of variable argument functions, and removing overheads of
  type-classes~\cite{wadler1989make,hall_type_1996,oliveira_type_2010}.

\end{itemize}

We have implemented a staging extension for Scala \tool \footnotemark[3].
 \tool has a minimal interface (\sct{sct:interface}) based on type annotations.
 We have evaluated performance gains and the validity of \tool on all case
 studies~(\sct{sct:case-studies}) and compared them to LMS. In all benchmarks (\sct{sct:evaluation})
 our evaluator performs the same as LMS and gives significant performance gains compared to original programs.

\footnotetext[3]{Source code: \url{https://github.com/scala-ct/scala-ct}.}
