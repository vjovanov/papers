\section{Related Work}
\label{sct:related-work}

% Staging LMS Type Directed partial evaluation
MetaOCaml~\cite{taha_multi-stage_1997,calcagno2003implementing} is a staging extension
 for OCaml. It uses quotation to determine the stage in which the term is executed. Types of quoted terms are annotated
 to assure cross-stage persistence. Staging in MetaOCaml starts at host language runtime and
 can not express compile-time computations. Further, operations on annotated types
 do not get automatically promoted to the adequate stage of computation as with compile-time views.
 Finally, there are no implicit conversions so all stage promotions of terms must be explicit.

% MacroML
MacroML~\cite{ganz2001macros} is a language that translates macros into MetaML staging executed
 at compile time to provide a ``clean'' solution for macros. In MacroML, within
 the \code{let mac} construct function parameters can be annotated as an early stage computation. These parameters
 can then be used in escaped terms, \ie, terms scheduled to execute at compile time. Unlike \tool, MacroML
 uses escapes and early parameters to mark terms scheduled for to execute at compile time. Within
 escapes terms scheduled for runtime again need to be marked with brackets. This kind of dual annotations are
 not required as compile-time views are automatically promoted to runtime terms.

% LMS Type Directed partial evaluation
In LMS~\cite{rompf2012lightweight} terms that are annotated with \code{Rep} types will be executed at
 the stage after runtime compilation. Therefore, LMS can not directly be used for compile time computation. Furthermore,
 LMS requires additional reification logic and code generation for all \code{Rep} types.

 Programming language Idris~\cite{brady2010scrapping} introduces the \code{static} annotation
  on function parameters to achieve partial evaluation. Annotation \code{static} denotes
  that the term is statically known and that all operations on that term should
  be executed at compile-time. However, since \code{static} is placed on terms rather
  then types, it can mark only \emph{whole terms} as static. This restricts the number
  of programs that can be expressed, \eg, we could not express that vectors in the
  signature of \emph{dot} are static only in size. Finally, information about \code{static}
  terms can not be propagated through return types of functions so \code{static}
  in Idris is a partial evaluation construct, i.e., it hints that partial evaluation
  should be applied if function arguments are static.

% Specialization Scenarios
% TODO should we \cite{le2004specialization}

% Hybrid Partial Evaluation
Hybrid Partial Evaluation~(HPE)~\cite{shali2011Hybrid} is a technique for partial evaluation that
 does not perform binding time analysis (similarly to online partial evaluators) but relies on the user
 provided annotation \code{CT}\footnote{Name \code{ct} in \tool is inspired by hybrid partial evaluation.}.
 HPE implementations exist for both Java and Scala~\cite{sherwany2015refactoring}.
 Although, \code{CT} is used for partial evaluation, it does not affect typing of user programs. Furthermore,
 behavior of \code{CT} in context of generics is not described. \tool can be seen
 as statically typed version of hybrid partial evaluation with support for parametric polymorphism.
 Due to the support for parametric polymorphism \tool can express compile-time data structures with
 dynamic data.

% Forge
Forge~\cite{forge}, by Sujeeth et al., uses a DSL to declare a specification of the libraries.
 Forge then generates both unannotated and annotated code based on the specification.
 Their language also supports generating staged code (comprised of terms different from multiple stages).
 Forge specification and code generation supports only a subset of Scala guided towards the
 Delite~\cite{brown_heterogeneous_2011,composition-ecoop2013} framework.

% Yin-Yang
The Yin-Yang framework, by Jovanovic et al.~\cite{yin-yang}, solves the problem
 of code duplication by generating reification and code generation logic based on Scala code of existing types.
 With their approach there is no code duplication for the supported language features. However, not all of the
 Scala language is supported and all generated terms are generated for the next stage, thus,
 making a stage distinction is impossible.
