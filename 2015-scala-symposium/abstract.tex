In staged languages each term is executed at a single \emph{compilation stage}
 that is determined based on: quotation (e.g., MetaOCaml), or types (e.g., LMS).
 To ensure \emph{cross-stage persistence} (i.e., correctness) types of next/previous
 stage terms must annotated. Consequentially staged functions are \emph{monovariant}:
 their arguments can not vary with respect to a compilation stage.

Type based staged languages require type annotations of all \emph{next stage}
 terms, as well as implementing reification and code generation logic for all
 next stage types. Further, when we use staging at host language compile-time,
 all terms scheduled to execute at runtime require type annotations and all
 libraries used at runtime require logic for reification and code generation.
 Annotations are especially noticeable in languages with local type
 inference as types of method parameters must be provided by the programmer.

We introduce a type based staging extension for Scala where we transparently
 extend Scala with polymorphic binding-time information. By using this information
 we enable parametrically polymorphic parameters to also be stage polymorphic.
 With our scheme defining stage polyvariant functions can be done through
 existing language features (i.e., bounded parametric polymorphism).  Further,
 terms whose types are monomorphically annotated are executed in the earlier
 stage of compilation, in our case, at host language compile-time. Annotated
 types rely on the polymorphic binding time to represent merely \emph{\mbox
 {compile-time} views} of original types. This obviates the necessity for
 reification and code generation logic. We compare our framework with LMS and
 show that in  majority of programs we require less type annotations while we
 achieve same  performance  improvements.
