\documentclass{llncs}
\usepackage{llncsdoc}

% Build
\usepackage{subfiles}

% Math
\usepackage{amsmath}
\usepackage{amssymb}
\usepackage{math}
\usepackage{mathtools}

% Graphics
\usepackage{graphicx}

% Text stuff
\usepackage{listings}
%% sstucki: comment this back in later. my eyes hurt from the
%% pixelated substituted fonts...
%\usepackage[T1]{fontenc}
\usepackage[utf8]{inputenc}
\usepackage{url}
\usepackage{hyperref}
\usepackage[usenames]{color}
\usepackage[font=bf,labelfont=bf]{caption}
\DeclareCaptionType{copyrightbox}
\usepackage{subcaption}
\usepackage{multirow}
\usepackage{etoolbox}
\usepackage{stmaryrd}

% PL Formulas
\usepackage{fleqn}
\usepackage{latexsym}
\usepackage{bcprules}
\usepackage{prooftree}

\patchcmd{\maketitle}{\@copyrightspace}{}{}{}
%\usepackage[font=bf,labelfont=bf]{caption}

%for Strikethrough, to remove when not required anymore
\usepackage[normalem]{ulem}

% Layout
\usepackage{xspace}% space in macros
\usepackage{multicol}

% Makes tables look beautiful
\usepackage{booktabs}
\usepackage{tabularx}

\input{yy_macros}

% ----- begin macros
\lstdefinelanguage{Scala}%
{morekeywords={abstract,%
  case,catch,char,class,%
  def,else,extends,final,for,%
  if,import,implicit,%
  match,module,%
  new,null,%
  object,override,%
  %package,% commented out for a specific example
  private,protected,public,%
  for,public,return,super,%
  this,throw,trait,try,type,%
  val,var,%
  with,while,%
  yield,%
  macro%
  },%
  sensitive,%
  morecomment=[l]//,%
  morecomment=[s]{/*}{*/},%
  morestring=[b]",%
  morestring=[b]',%
  showstringspaces=false%
}[keywords,comments,strings]%

\lstset{language=Scala,%
  mathescape=true,%
%  columns=[c]fixed,%
%  basewidth={0.5em, 0.40em},%
  aboveskip=1pt,%\smallskipamount,
  belowskip=1pt,%\negsmallskipamount,
  lineskip=-0.2pt,
  basewidth={0.54em, 0.4em},%
%  basicstyle=\ttfamily,%\scriptsize,%
  basicstyle=\footnotesize\ttfamily,
  keywordstyle=\sffamily\bfseries%
%  keywordstyle=\sffamily\bfseries,%
%  xleftmargin=0.5cm
}


\newcommand{\commentstyle}[1]{\slseries{#1}}
\newcommand{\keywordstyle}[1]{\bfseries{#1}}

% Code
\lstnewenvironment{listing}{\lstset{language=Scala}}{}
\lstnewenvironment{listingsmall}{\lstset{language=Scala,basicstyle=\small\ttfamily}}{}
\lstnewenvironment{listingtiny}{\lstset{language=Scala,basicstyle=\scriptsize\ttfamily}}{}

\newcommand{\scode}[1]{\lstinline[language=Scala,columns=fixed,basicstyle=\ttfamily,keywordstyle=\ttfamily]|#1|}
\newcommand{\jcode}[1]{\lstinline[language=Java,flexiblecolumns=true,basicstyle=\ttfamily]{#1}}

\newcommand{\code}[1]{\scode{#1}}
\newcommand{\sct}[1]{(\S \ref{#1})}

% TODOs:
\newif\ifshowTodos
\showTodostrue
\showTodosfalse % Uncomment this to hide all TODOs.
\ifshowTodos
\newcommand{\todo}[1]{{\color{red} \textbf{[TODO: #1 ]}}}
\newcommand{\sstucki}[1]{{\color{red} [\textbf{sstucki}: #1 ]}}
\newcommand{\manojo}[1]{{\color{red} [\textbf{manojo}: #1 ]}}
\newcommand{\comm}[1]{}
\newcommand{\orElse}[2]{{\color{red} [\sout{#1} {\color{magenta} #2}]}}
\else
\newcommand{\todo}[1]{}
\newcommand{\sstucki}[1]{}
\newcommand{\manojo}[1]{}
\newcommand{\comm}[1]{}
\newcommand{\orElse}[2]{#2}
\renewcommand{\sout}[1]{}
\fi
% paper specific commands
\newcommand{\tool}{Yin-Yang\xspace}



\begin{document}

\title{Yin-Yang: Concealing the Deep Embedding of DSLs}
\subtitle{Demonstration: a DSL for in-memory querying}

\author{Vojin Jovanovic, Amir Shaikhha, and Manohar Jonnalagedda}

\institute{EPFL, Switzerland\\
\email{\{first\}.\{last\}@epfl.ch}}

\maketitle

\begin{abstract}
Deeply embedded domain-specific languages (EDSLs) inherently compromise programmer experience for improved program performance. Shallow EDSLs complement them by trading program performance for good programmer experience. We present Yin-Yang, a framework for DSL embedding that uses Scala reflection to reliably translate shallow EDSL programs to their corresponding deep EDSL programs. This automatic \emph{translation} allows program prototyping and debugging in the user friendly shallow embedding, while the corresponding deep embedding is used in production\textemdash where performance is important. The reliability of the translation completely conceals the deep embedding from the DSL user. For the DSL author, Yin-Yang generates deep DSL embeddings from their shallow counterparts by reusing the program translation. This obviates the need for code duplication and assures that the implementations of the two embeddings are always synchronized.
\end{abstract}
% \category{D.3.3}{Programming Languages}{Language Constructs and Features}

\keywords
Embedded Domain-Specific Languages, Macros, Deep Embedding, Shallow Embedding, Compile-Time Meta-Programming

\section{Introduction}

% External DSLs
External Domain-specific languages (DSLs) are languages with a custom compiler specialized to a particular application domain. Knowledge about the domain and specialization of the language give the compiler possibilities for optimization that do not exist in \emph{general-purpose languages}. A restricted language makes it easy for \emph{DSL users} to learn the language while the optimizations can yield execution times close to hand-optimized programs \cite{rompf_optimizing_2013}.

% Embedded DSLs
The development of external DSLs includes building a parser and a type checker as well as a large tool-chain (i.e., debuggers, IDEs, and documentation tools). An appealing alternative are \emph{embedded DSLs}~(EDSLs)~\cite{Hudak96csur} that reuse the parser, the type checker, and the tool-chain of a general-purpose host language to minimize required development effort. We can classify DSL embeddings into:

\begin{itemize}

\item \emph{Shallow embeddings.} Values in the embedded language are \emph{directly} represented by the values in the host language. A special sub-category of shallow embeddings are \emph{direct embeddings} where language constructs and term types are exactly the same as in the host language.

\item \emph{Deep embeddings.}  Values in the embedded language are symbolically (with data structures) represented in the host language.

\end{itemize}

Direct embeddings are typically friendly for the \emph{DSL user} as they \emph{i)} \emph{linguistically match} the host language and \emph{ii)} can be easily debugged since values of the embedded language directly correspond to the values of the host language. However, since the structure of the programs is not completely known, the number of possible domain-specific optimizations is limited.

% Deep embedding exhibits good performance but interface suffers
Deep embeddings reify the DSL programs into an \emph{intermediate representation} (IR). This reification hinders usability as it often relies on complex type system constructs that create a linguistic mismatch with the host language. Furthermore, the IR construction in the programs disallows value inspection with classical debuggers and makes the problem even harder. However, the IR being domain specific opens new opportunities for optimization.


% Direct interface with meta-programming.
The main idea of \tool is to use \emph{reflection} to translate programs
written in an unmodified direct embedding into their deeply embedded
counterparts.  Since the fundamental difference between the interfaces of the
two embeddings is in their types, we propose a generic translation between
the two embeddings that is configurable with a separate \emph{type
translation}.

% What does Yin-Yang do?
The translation forms the core of \tool, a generic framework for DSL
embedding, that uses Scala's macros~\cite{burmako_scala_2013} to reliably
translate direct \edsl{} programs into the corresponding deep \edsl programs.
The virtues of the direct embedding are used during program development when
performance is not important. The translation is then applied when performance
is essential or alternative interpretations of a program are required. In
effect, the translation completely conceals the deep embedding from the user.

To avoid error prone maintenance of synchronized direct and deep embeddings
\tool reuses the core translation to generate the deep embeddings based on the
definition of direct embeddings. Since the same translation is applied both
for the \edsl definition and the \edsl program, the equivalence between the
embeddings is assured. With deep embedding generation DSL author can focus on
the domain-specific optimizations while the interface is completely handled by
\tool.


\section{Demonstration: A DSL for In-Memory Queries}

  

  % We generate deep embedding for the shallow embedding
  Our demonstration shows how \tool is used by: \emph{i)} a DSL author and \emph{ii)} a DSL user. The demonstration shows how the DSL user can be agnostic about the deep embedding. Furthermore, the DSL author can program deep embeddings and program transformations just by using the syntax of the direct embedding. For the purpose of the demonstration we use a DSL for writing in-memory queries. The demonstration proceeds as follows: 

\lstset{mathescape=false}
\begin{figure*}[ht]
\begin{multicols}{2}
\begin{subfigure}[b]{1\linewidth}
\centering

\begin{listingtiny}
class SelectOp[A](parent: Oper[A])
  (pred: A => Boolean) extends Oper[A]{
  def open() = parent.open
  def next() = parent findFirst pred    
  def reset() = parent.reset
  def close() = ()
}
\end{listingtiny}
\label{lst:shallow}
\end{subfigure}

\begin{subfigure}[b]{1\linewidth}
\begin{listingtiny}
def transform(s: Sym[Any]): Exp[Any] = s match {
  case sql"join($r, join($s, $t))" => 
    ql"join(join($r, $s), $t)"
  case sql"join(join($r, $s), $t)" => 
    ql"join($r, join($s, $t))"
  case _ => super.transform(s)
}
\end{listingtiny}

\end{subfigure}
\end{multicols}
\caption{\label{lst:select} A selection query operator (left) and the rules for \code{join} reordering (right).}
\end{figure*}

  % We will write operators for the query language in the direct embeddings
  {\bf DSL author}: Writes a na\"{\i}ve Scala implementation of common query operators. An example operator is presented in the left part of \figref{lst:select}.

  {\bf DSL user}: Can immediately write concise queries in the DSL. Although the query performance is satisfactory for prototyping, in production, the overhead is unacceptably large.

  % From the direct embedding we will generate the deep embedding
  {\bf DSL author}: To improve on the situation the author decides to provide a deep embedding for the query engine. All he needs to do is place a single Scala class annotation per query operator he implemented. The annotation marks that the annotated operator we would like to have the deep embedding generated. \tool then processes the annotated classes and generates the deep embedding that includes reification logic and code generation.

  % The DSL user will write queries.
  {\bf DSL user}: Places queries in a DSL scope. After putting queries into the DSL scopes the user is still faced with the same type errors as before. The program execution errors can still be debugged in a standard debugger. In production, the DSL scope will invoke the translation and apply all standard compiler optimizations to the DSL yielding significant performance improvements.

  % Demonstrate the type errors, debugging, and prototyping.
  {\bf DSL author}: Decides to further improve performance by introducing the domain specific optimizations in the query language. The DSL author declares simple rewrite rules that the DSL compiler will later apply. The rewrite rules are specified through Scala quasi-quotes~\cite{shabalin2013quasiquotes} and thus the DSL author can stay completely agnostic of the underlying data structures. The rules for \code{join} associativity are presented in the right part of \figref{lst:select}.
  
  % Performance
  {\bf DSL user}: Then measures performance of his DSL and realizes that the generated code performs on par with the hand-optimized version of the query.

\bibliographystyle{plain}

\bibliography{vjovanov-lib,manual}

\end{document}