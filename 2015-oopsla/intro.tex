\section{Introduction}
\label{sct:introduction}

% TODO count lines of code in LMS based DSLs (hyper) where stage distinction is relevant.
% TODO better citations metaOCaml, interpreters to compilers, DSLs
 Multi-stage programming (or \emph{staging}) is a flavor of meta-programming
  where compilation is separated in multiple \emph{stages}. Execution of each
  stage outputs code that is executed in the next stage of compilation. The first
  stage of compilation is the \emph{host language} compile-time, the second stage
  the host language runtime, the third stage is the runtime of run-time generated
  code, etc. Notable, staging frameworks are MetaOCaml~\cite{taha_multi-stage_1997}
  and LMS~\cite{rompf2012lightweight}, and they were successfully applied as a
  \emph{partial evaluatior}~\cite{jones1993partial}: for removing abstraction
  overheads in high-level programs~\cite{carette2005multi,rompf2012lightweight},
  for domain-specific languages~\cite{jonnalagedda2014staged}, and for converting language
  interpreters into compilers~\cite{lancet,futamura1999partial}.

 We show an example of how staging is used for partial evaluation of a function
 for computing the inner product of two vectors~\footnotemark[1]:\begin{lstparagraph}
def dot[V:Numeric](v1: Vector[V], v2: Vector[V]): V =
  (v1 zip v2).foldLeft(zero[V]) {
    case (prod, (cl, cr)) => prod + cl * cr
  }
 \end{lstparagraph}

In function \code{dot}, if the vector sizes are static during program runtime the inner product can
 be partially evaluated into a sum of products of vector components. To achieve partial evaluation,
 we must communicate to the staging framework that the operations on values of vector components
 should be executed in the next stage (after language runtime). The compilation stage
 in which a term will be executed is determined by \emph{code quotation}~(in MetaOCaml)
 or by specific parametric types \code{Rep}~(in LMS). In LMS~\footnotemark[2] we mark
 that the vector size is statically known by annotating only vector elements with
 a \code{Rep} type:\begin{lstparagraph}
def dot[V:Numeric]
  (v1: Vector[Rep[V]], v2: Vector[Rep[V]]): Rep[V]
 \end{lstparagraph}

Here the \code{Rep} annotations on \code{V} denote that elements of vectors will be known
 only in the next stage (after run-time compilation). At this stage the \code{zip}
 \code{foldLeft}, and pattern matching inside the closure will not exist
 as they were evaluated at the previous stage (host language runtime). Note that
 the unquoted/unannotated code is always executed during host-language runtime
 and quoted/type annotated code is executed after run-time compilation.

{\bf First Problem.} How can we use staging for programs whose values are statically
 known at the host language compile-time (the first stage)? All staging frameworks
 treat unannotated terms as host language runtime values and annotated terms as
 values of later stages. Even if we would start staging one stage earlier (at host language compile-time),
 we would have to annotate all run-time values. Annotating all values is cumbersome
 since host language run-time values comprise the majority of user programs~\todo{cf.}.

\footnotetext[1]{All code examples are written in \emph{Scala}. For comprehension
of the paper basic knowledge of Scala is necessary.}
\footnotetext[2]{In this work we use LMS as it is the only staging framework in Scala.}

Programming languages Idris and D allow try to solve this problem by allowing
 the \code{static} annotation on function arguments. Annotation \code{static} denotes
 that the term is statically known and that all operations on that term should
 be executed at compile-time. However, since \code{static} is placed on terms rather
 then types, it can mark only \emph{whole terms} as static. This restricts the number
 of programs that can be expressed, \eg, we could not express that vectors in the
 signature of \emph{dot} are static only in size. Finally, information about \code{static}
 terms can not be propagated through return types of functions, so \code{static}
 in Idris and D is more a partial evaluation construct.

{\bf Second Problem.} Staging systems based on type annotations (e.g., LMS and type-directed
partial evaluation~\cite{danvy1999type}) inherently require duplication of code as
,a priory, no operations are defined on \code{Rep} annotated types. For example,
in the \code{dot} function all numerical types (\eg, \code{Rep[Int]}, \code{Rep[Double]}, etc.)
must be re-implemented in order to typecheck the programs and achieve code generation
for the next stage.

Suereth et al.~\cite{forge} and Jovanovic et al.~\cite{yin-yang}
 propose generating code for the next stage computations based on
 a language specification. These approaches solve the problem,
 but they require writing additional specification for the libraries,
 require a large machinery for code generation,
 and support only restricted parts of Scala.

The main idea of this paper is that annotated types should denote computations
 that happen at the \emph{previous stage} instead of the next stage.
 The reason is two-fold: \emph{i)} annotating code of previous stages succinctly
 express compile-time execution and \emph{ii)} in staged programs the static terms
 appear less frequently than run-time terms, and in order to bear minimum overhead
 for the users, it is better to add annotation overhead to static terms.

Further, annotated types are simply a\emph{compile-time view}
 of existing data types and therefore no code duplication is necessary.
 The compile-time view makes all operations and non-generic fields executed in
 the host language compile time. The compile-time view requires programmers to
 define a single definition of a type. Then, the existing types can be promoted
 to their compile-time duals with the \code{@ct} annotation at the type level,
 and with the \code{ct} function on the term level. Consequently, due to the
 integration with the type system, the control over staging is
 fine-grained and polymorphic, and term level promotions obviate code duplication
 for static data structures.

% Restricted to compile-time, but just techincal.
With compile-time views, to require that vectors \code{v1} and \code{v2} are
 static and to partially evaluate the function, a programmer would need to make
 a simple modification of the \code{dot} signature:\begin{lstparagraph}
def dot[V: Numeric]
  (v1: Vector[V]@ct, v2: Vector[V]@ct): V
\end{lstparagraph}

This, in effect, requires that only vector arguments (not their elements) are
 statically known and that all operations on vector arguments will be executed
 at compile time (partially evaluated). Since, values are polymorphic the result
 of the function will either be a dynamic value, static value, or a compile-time value
 that can be further used for partial evaluation. Residual programs of calling \code{dot}
 with arguments from different stages:

\vspace{1.8mm}
\begin{listing}[mathescape]
  // [el1, el2, el3, el4] are dynamic decimals
  dot(ct(Vector)(el1, el2), ct(Vector)(el3, el4))
    $\hookrightarrow$ (el1 * el3 + el2 * el4): Double@dynamic

  dot(ct(Vector)(2.0, 4.0), ct(Vector)(1.0, 10.0))
    $\hookrightarrow$ (2.0 * 1.0 + 4.0 * 10.0): Double@static

  // ct promotes static terms to compile-time
  dot(ct(Vector)(ct(2), ct(4)),
      ct(Vector)(ct(1), ct(10)))
    $\hookrightarrow$ 42: Double@ct
\end{listing}
\vspace{1.8mm}

In this paper we contribute to the state of the art:
\begin{itemize}

 \item By introducing compile-time views as means to: \emph{i)} succinctly achieve staging
 at host language compile-time and to \emph{ii)} avoid code duplication in type based
 staging systems.

 \item By introducing the \calculus calculus (\sct{sct:calculus}) that in a
  fine-grained way captures the user's intent about partial evaluation. The calculus
  is based on $F_{<:}$ with lazy records which makes it suitable for representing
  modern multi-paradigm languages with object oriented features. Finally,
  we formally define evaluation rules for \calculus.

 \item By providing a \emph{translation scheme} from data types in object oriented languages
  (polymorphic classes and methods) into their dual compile-time views in the
  \calculus calculus (\sct{sct:scala-translation}).

 \item By demonstrating the usefulness of compile-time views in four case
 studies (\sct{sct:case-studies}): inlining, partially evaluating recursion,
 removing overheads of variable argument functions, and removing overheads of
 type-classes~\cite{oliveira2010type}.

\end{itemize}

We have implemented \tool according to the translation
 scheme~(\sct{sct:scala-translation}) from object oriented features of Scala to
 the \calculus calculus. The prototype implemented for Scala and
 open-sourced\footnotemark[3]. It has a minimal Scala
 interface (\sct{sct:interface}) based on type annotations. We have evaluated
 performance gains and the validity of the partial evaluator on all case
 studies~(\sct{sct:case-studies}) and compared them to LMS. In all benchmarks
 our evaluator gives significant performance gains compared to original programs and
 performs equivalently to LMS.
\footnotetext[3]{Source code: \url{https://github.com/scala-inline/scala-inline}.}
