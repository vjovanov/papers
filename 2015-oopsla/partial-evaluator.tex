\section{Compile-Time Views in Scala}
\label{sct:interface}

% TODO This is all together not good without having @inline at the term level (classes,
% objects, traits, methods, etc.). Unfortunately, this opens up a new can of worms
% that we are not ready for now. If the paper is ready by Tuesday evening we can take a
% stab at it.

% Motivate: data structure that whenever instantiated is already ct.
% Example: complex number, parser ...
% inline on classes and their methods
% inline on traits
% inline and inheritance

We have implemented \tool, a staging extension for Scala based on compile-time views.
 \tool is a compiler plugin that executes in a phase after the
 Scala type checker. The plugin starts with pre-typed Scala programs and uses
 type annotations~\cite{odersky_1996_putting} to track and verify information about the biding-time
 of terms. Currently, it supports only two stages of compilation: host language compile-time
 (types annotated with \code{@ct}) and host language run-time (unannotated code).

To the user, \tool exposes a minimal interface (\figref{fig:interface}) with
annotations \code{inline} and \code{ct}, and functions \code{inline} and \code{ct}.

\begin{figure}
\begin{listing}
package object scalact {

  final class ct extends StaticAnnotation
  final class inline extends StaticAnnotation

  @compileTimeOnly def ct[T](body: => T): T = ???
  @compileTimeOnly def inline[T](body: => T): T = ???

}
\end{listing}
\label{fig:interface}
\caption{Interface of the \tool.}
\end{figure}

\smartparagraph{Annotation \code{ct}} is used at the type level~(\eg,
\code{Int@ct}) and denotes a compile-time view of a type. The
annotation is integrated in the Scala's type system  and, therefore, can be
arbitrarily nested in different variants of types. Table  \ref{tbl:ct-type}
shows how the \code{@ct} annotation can be placed on types and how it, due to
the translation to the compile-time views (\figref{ct-translation}), changes
method signatures on annotated types.

\begin{table*}[t]
\caption{Types and corresponding method signatures after translation to their compile-time views.}
\label{tbl:ct-type}
\centering
\begin{tabularx}{\linewidth}{ X X X X }
\toprule

  Annotated Type              & \ &  Type's Method Signatures                          &  \\
  \code{Int@ct}               & \ &  \code{+(rhs: Int@ct): Int@ct}                     &  \\
  \code{Vector[Int]@ct}       & \ &  \code{map[U](f: (Int => U)@ct): Vector[U]@ct}     &  \\
                              & \ &  \code{length: Int@ct}                             &  \\
  \code{Vector[Int@ct]@ct}    & \ &  \code{map[U](f: (Int@ct => U)@ct): Vector[U]@ct}  &  \\
                              & \ &  \code{length: Int@ct}                             &  \\
  \code{Map[Int@ct, Int]@ct}  & \ &  \code{get(key: Int@ct): Option[Int]@ct}           &  \\

\bottomrule
\end{tabularx}
\end{table*}

 In \tabref{tbl:ct-type}, \code{Int@ct} is a non-polymorphic type and therefore
 according to the translation to the compile-time view (\ref{fig:ct-translation})
 parameters of all methods will also be compile-time views. On the other hand,
 \code{Vector[Int]@ct} will have parameters of all methods transformed except
 the generic ones. In effect, this, makes higher order combinators of \code{Vector}
 operate on dynamic values, thus, function \code{f} passed to \code{map} accepts
 the dynamic value as input. Type \code{Vector[Int@ct]@ct} has all parts executed
 at compile-time. The return type of the function \code{map} can still be
 both dynamic and a compile-time view: due to the type parameter \code{U}.

\smartparagraph{Annotation inline} can be used only at the term level on statically
 known methods and functions. It denotes that the method/function will be inlined during
 compilation time. In other words, \code{inline} is marking that the function application
 is a compile-time computation and that application should be removed by partial evaluation.
 This is not the first time that inlining is achieved through partial
 evaluation~\cite{monnier2003inlining}.

 Internally \code{inline} can be expressed in terms of the \code{ct} annotation. A method\begin{lstparagraph}
@inline def dot[V: Numeric]
  (v1: Vector[V], v2: Vector[V]): V
\end{lstparagraph}
  will have an internal method type\begin{lstparagraph}
((v1: Vector[V], v2: Vector[V]) => V)@ct
\end{lstparagraph} that can not be written by the users. We choose the name
 \code{inline} to be consistent with the existing Scala \code{inline} annotation.


\smartparagraph{Functions \code{ct} and \code{inline}} are used at the term level
 for promoting Scala objects and methods/functions into their compile-time views. Without the
 the \code{ct} and \code{inline} we would not be able to instantiate compile-time views of types.
 \tabref{tbl:ct-type} shows how different types of terms are promoted to their
 compile-time views.

\begin{table*}[t]
\caption{Promotion of terms to their compile-time views.}
\label{tbl:ct-type}
\centering
\begin{tabularx}{\linewidth}{ X X }
\toprule

  Promoted Term        \quad \quad \quad & Term's Promoted Type             \\
  \code{ct(Vector)(1, 2, 3)            } & \code{: Vector[Int]@ct        }  \\
  \code{ct(Vector)(ct(1), ct(2), ct(3))} & \code{: Vector[Int@ct]@ct     }  \\
  \code{new (Cons@ct)(1, Nil)          } & \code{: Cons[Int]@ct          }  \\
  \code{new (Cons@ct)(ct(1), ct(Nil))  } & \code{: Cons[Int@ct]@ct       }  \\
  \code{ct((x: Int) => x)              } & \code{: (Int@ct => Int@ct)@ct }  \\
  \code{inline((x: Int) => x)          } & \code{: (Int => Int)@ct       }  \\

\bottomrule
\end{tabularx}
\end{table*}

Function \code{ct} can be applied to objects (\eg, \code{Vector}) to provide a compile-time
 view over their methods. When those objects have generic parameters, \code{ct} be used
 to promote the arguments, and thus, the result types of these functions. When applied,
 on functions \code{ct} promotes the compile-time view as well as its arguments
 and the return type.

Function \code{inline} can be applied on functions/methods to promote only the function/method
 to their compile time views without promoting the arguments. This function can be
 seen as a shallow version of \code{ct} that makes only the outer type a compile-time view.

\subsection{Tracking Binding-Time of Terms}
\label{sct:static}


 Internally \tool has additional type annotations for tracking the binding-time of terms.
  Type of each term is annotated with either \code{dynamic}, \code{static}, or \code{ct}. \code{dynamic} denotes
  that the term can only be known at runtime, \code{static} that the term is known
  at compile-time but it will not be computed at compile time, and \code{ct} that
  the term will be computed at compile-time.
 % TODO Cite better the tracking of static terms in static languages (polymorphic version e.g.)
 Tracking static terms was studied in the context of binding-time analysis
  in partial evaluation for typed~\cite{nielson_1988_automatic} and
  untyped~\cite{gomard1991partial} languages. We use similar techniques
  (described in~\sct{sct:calculus}), however, unlike in partial evaluation we
  do not evaluate static terms at compile time. Static terms are evaluated only
  when they are explicitly marked by the programmer with \code{ct}.

  % What are the static terms
\subsection{Least Upper Bounds}
\label{sct:lub}

 We use subtyping of Scala to simplify tracking of binding times by introducing a
 subtyping relation between \code{dynamic}, \code{static}, and \code{ct}. We argue that
 a \code{static} type is a more specific \code{dynamic} as it is statically known
 and that \code{ct} is more specific than \code{static} as its operations are executed
 at compile time. Therefore we establish that\begin{lstparagraph}
                 ct <: static <: dynamic
\end{lstparagraph}

 The use of subtyping simplifies verification of validty of function calls and helps computing the
 least upper bounds of terms. For example, validity of function calls:\begin{lstparagraph}
ct(List)(1, ct(2)): List[Int@static]@ct
ct(List)(ct(1), ct(2)): List[Int@ct]@ct
ct(List)((x: Int@dynamic), ct(2)): List[Int@dynamic]@ct
\end{lstparagraph}

 % todo cite literature (who else does this)
 Notable exception are control flow constructs for which the original Scala least
 upper bound rules do not hold. The binding-time of control flow constructs does not
 depend only on return type of the body but also the conditional~\cite{}. For example, if
 both branches of an \code{if} construct are \code{static} the result can still be \code{dynamic}
 if the condition is \code{dynamic}. Here subtyping also helps as the binding type of the
 return value is simply calculated as \code{lub(c, thn, elz)} where \code{lub} is a function
 for computing least upper bounds, and \code{c}, \code{thn}, \code{elz} are respectively
 binding times of the condition, the then branch, and the else branch.

\subsection{Well-Formedness of Compile-Time Views}
\label{sct:wf-ctv}

% Nice description of csp and pointer to the right paper.
Earlier stages of computation can not depend on values from later stages. This property,
 defined as \emph{cross-stage persistence}~\cite{taha_multi-stage_1997,westbrook2010mint},
 imposes that all operations on compile-time views must known at compile time.

To satisfy cross-stage persistence \tool verifies that composite dynamic
 types~(\eg, polymorphic-types, function types, record types, etc.) are not composed
 of compile-time views. The intuition is that all method parameters~(including \code{this})
 of compile time views must either be a compile-time view or them selves type variables. In the following example,
 we show malformed types and examples of terms that are inconsistent with causality\begin{lstparagraph}
xs: List[Int@ct]     => ct(Predef).println(xs.head)
fn: (Int@ct=>Int@ct) => ct(Predef).println(fn(ct(1)))
\end{lstparagraph}

In the first example the program should print the head of the dynamically known list
 at compile time. In the second example the statement should print the result of \code{fn} at
 compile time but the body of the function is unknown.

% Examples on classes
The \code{inline} annotation promotes only function/method bodies to compile-time views. In effect,
 this requires only the method/function body to be known at compile time. Method bodies
 are statically known in objects and classes with final methods, thus, the \code{inline}
 annotation is only applicable on such methods.

\subsection{Convenient Implicit Conversions}
\label{sct:implicits}

% Requires desugaring
If method parameters require compile-time views of a type the corresponding arguments
 in method application would always have to be promoted to \code{ct}.
 In practice this is not convenient as it requires an inconveniently large number
 of annotations. Staging is commonly used for optimization of libraries, and as such, it should not
 affect user code - users should not be aware of the internal operation of the library.

To address this issue we introduce implicit conversions from \code{static} terms to \code{ct} terms.
 The conversions support translation of language literals, direct class constructor calls with static arguments, and static method
 calls with static arguments into their compile-time views. Since our compile-time evaluator does
 not use Asai's~\cite{asai2002binding,sumii2001hybrid} method to keep track of
 the value of each static term, we dissalow implicit conversions of terms with static variables.

For example, for a factorial function \begin{lstparagraph}
def fact(n: Int @ct) = if (n == 0) 1 else fact(n - 1)
 \end{lstparagraph} we will not require annotations on literals \code{0}, and \code{1}. Furthermore,
 the function can be invoked without promoting the literal \code{5} into it's compile-time view:\begin{lstparagraph}
fact(5)
  $\hookrightarrow$ 120
 \end{lstparagraph}
